\chapter{Preintegrated sensors}
\minitoc

\cite{buchanan2021learning} -> Learning Inertial Odometry for Dynamic Legged Robot State Estimation


\section{Motion integration}
First, as a motivational example, we will look at a simple 2D integration motion integration problem. 
Let's define the state of our robot as the pose $\T{w}{b} \in \SE (2) or \SE(3)$. 
Let's assume we have a source of odometry, coming from wheel encoders for instance, providing velocity measurements in the base frame 
$\bfz_k = [\vel{b}{}, \angvel{b}{}]_k$. Now, to update our current estimate of the state, we just have integrate during dt (one step) 
using the matrix exponential, as seen in ..., to create a intantaneous relative transformation $\delta_k$ and then compose it with our previous step:

\begin{equation}
    \begin{split}
        &\delta_k = \Exp(\bfz_k * dt) 
        \\
        &\T{}{k} = \T{}{k} \delta_k
    \end{split}
\end{equation}

Now, assume that we a set of such measurements between 2 \KFs $i$j and $j$ $\mathcal{Z_{ij}} = [\tilde\bfz_k]_{i..j}$. 
For high frequency sensors, integration is costly. We want to do it once and for all.
We need to integrate the measurements the two \KFs timestamp in order to obtain a single measurement and derive a factor out of it.
In this example, this is simply done by composing the $\delta_k$ recursively one at a time.

\begin{equation}
    \T{}{j} = \T{}{i} \prod_{k=i}{j}\delta_k = \T{}{i} \tilde\Delta_{ij}
\end{equation}

Then a residual can be created using the $\Log$ operator:

\begin{equation}
    \bfr(\T{}{i}, \T{}{j}) = \Log(\T{}{i}^{-1}\T{}{j} \tilde\Delta_{ij})
\end{equation}

However, in some cases, those velocity measurements are biased, due to a modelling error (different encoder wheel raidus). 
To obtain meaningful estimations, we need to model this bias and include it in our estimation problem. If we consider a simple additive bias in our example
$\tilde\bfz_k = \bfz_k + \bfb$, we see that the ensulting(?) $\tilde\Delta(\bfb)$ measurements depend on $\bfb$ in a non trivial way. Therefore,
each time a numerical comes up with a new estimate for $\bfb$, to compute the exact $\tilde\Delta(\bfb)$, the integration of $\mathcal{Z_{ij}}$ 
needs to be recomputed which is computationaly costly. As we will see in the next section, this is solved by linearizing $\tilde\Delta(\bfb)$ with respect to $\bfb$ around
the bias $\ol\bfb$ that was estimated at the time of the integration. 

On the other end, for other kind of sensors (such as IMUs), it is harder to make the distinction between a delta quantity only depending on raw measurement (and an occasional estimated bias).
We have therefore to find special coordinate systems in which to integrate measurements in order to obtain an efficient algorithm. To obtain a factor, we also need to consider integrate 
noise odometry noise at each step to obtain the total covariance of the integrated motion.

These foundational ideas introduced in \cite{lupton-09}
  

\section{Generalized preintegration}
\section{IMU preintegration}
\section{IMU preintegration on Lie groups}
\section{External force preintegration}