\chapter{Preintegrated sensors}
\minitoc

\cite{buchanan2021learning} -> Learning Inertial Odometry for Dynamic Legged Robot State Estimation


% \section{Motion integration}
% First, as a motivational example, we will look at a simple 2D integration motion integration problem. 
% Let's define the state of our robot as the pose $\T{w}{b} \in \SE (2) or \SE(3)$. 
% Let's assume we have a source of odometry, coming from wheel encoders for instance, providing velocity measurements in the base frame 
% $\bfz_k = [\vel{b}{}, \angvel{b}{}]_k$. Now, to update our current estimate of the state, we just have integrate during dt (one step) 
% using the matrix exponential, as seen in ..., to create a instantaneous relative transformation $\delta_k$ and then compose it with our previous step:

% \begin{equation}
%     \begin{split}
%         &\delta_k = \Exp(\bfz_k * dt) 
%         \\
%         &\T{}{k} = \T{}{k} \delta_k
%     \end{split}
% \end{equation}

% Now, assume that we a set of such measurements between 2 \KFs $i$j and $j$ $\mathcal{Z_{ij}} = [\tilde\bfz_k]_{i..j}$. 
% For high frequency sensors, integration is costly. We want to do it once and for all.
% We need to integrate the measurements the two \KFs timestamp in order to obtain a single measurement and derive a factor out of it.
% In this example, this is simply done by composing the $\delta_k$ recursively one at a time.

% \begin{equation}
%     \T{}{j} = \T{}{i} \prod_{k=i}{j}\delta_k = \T{}{i} \tilde\Delta_{ij}
% \end{equation}

% Then a residual can be created using the $\Log$ operator:

% \begin{equation}
%     \bfr(\T{}{i}, \T{}{j}) = \Log(\T{}{i}^{-1}\T{}{j} \tilde\Delta_{ij})
% \end{equation}

% However, in some cases, those velocity measurements are biased, due to a modeling error, for instance a wrong wheel radius. 
% To obtain meaningful estimations, we need to model this bias and include it in our estimation problem. If we consider a simple additive bias in our example
% $\tilde\bfz_k = \bfz_k + \bfb$, we see that the resulting $\tilde\Delta(\bfb)$ measurements depend on $\bfb$ in a non trivial way. Therefore,
% each time a numerical comes up with a new estimate for $\bfb$, to compute the exact $\tilde\Delta(\bfb)$, the integration of $\mathcal{Z_{ij}}$ 
% needs to be recomputed which is computationally costly. As we will see in the next section, this is solved by linearizing $\tilde\Delta(\bfb)$ with respect to $\bfb$ around
% the bias $\ol\bfb$ that was estimated at the time of the integration. 

% On the other end, for other kind of sensors (such as IMUs), it is harder to make the distinction between a delta quantity only depending on raw measurement (and an occasional estimated bias).
% We have therefore to find special coordinate systems in which to integrate measurements in order to obtain an efficient algorithm. To obtain a factor, we also need to consider integrate 
% noise odometry noise at each step to obtain the total covariance of the integrated motion.

% These foundational ideas introduced in \cite{lupton-09}
  
\section{A motivational example: IMU integration for graph optimization}
In this section we will introduce the IMU measurement model and describe a way in which we may naively integrate these measurements between \keyframes.
We will then explain the observation that lead to the development of the IMU preintegration algorithm in \cite{lupton-09}.

Let's consider the estimation of a robot base pose and velocity in an the world frame. Among available sensors, the IMU measurements are known to be noisy, 
biased and affected by the gravity. Without loss of generality, we assume that the IMU frame is identical to the base frame in the following developments.

\begin{equation}
    \begin{split}
    \angvelm{}{} &= \angvel{B}{WB} + \bias_{\angvel{}{}} + \noise_{\angvel{}{}} 
    \\
    \accm{}{}    &= \Rot{B}{W} \grav \acc{B}{WB} + \bias_{\acc{}{}} + \noise_{\acc{}{}} 
    \end{split}
    \label{eq:imu_meas_model}
\end{equation}
    
Other sensors can help estimate imu biases so we include them in the estimator state.

\begin{equation}
    \bfx = [\posi{W}{WB}, \vel{W}{WB}, \Rot{W}{B}, \bias_{\bfa}, \bias_{\angvel{}{}}]
    \triangleq 
    [\bfp, \bfv, \Rot{}{}, \bias_{\bfa}, \bias_{\angvel{}{}}] 
\end{equation}

Once the IMU has been started, these biases slowly change over time, depending on the quality of the IMU used. This change is almost always modeled 
as a random walk which is close to the observed behavior for rather short periods of time \footnote{Random walk tends stochastically to infinite values
while biases are observed to have bounded values}. A simple dynamical model based on strapdown integration of IMU measurements can then be derived.

\begin{equation}
    \begin{split}
    \dot{\posi{}{}} &= \vel{}{}  \\
    \dot{\vel{}{}} &= \acc{W}{WB} \\
    \dot{\Rot{}{}} &= \Rot{}{} [\angvel{B}{WB}]_{\times} \\
    \dot{\bias}_{\bfa} &= \noise_{\bfa}  \\
    \dot{\bias}_{\angvel{}{}} &= \noise_{\angvel{}{}} \\
    \end{split}
    \label{eq:imu_dyn_conti}
\end{equation}

Introducing the measurement equations \ref{eq:imu_meas_model} in continuous dynamics \ref{eq:imu_dyn_conti} and using a zero order hold
explicit Euler integration scheme results in the discrete dynamics:

\begin{equation}
    \begin{split}
    \Rot{}{}^{k+1}  &= \Rot{}{}^{k}Exp((\angvelm{}{}^k - \bias_{\angvel{}{}^k} - \noise_{\angvel{}{}^k})dt)
    \\
    \vel{}{}^{k+1}  &= \vel{}{}^{k} + \grav dt + \Rot{}{}^{k}(\accm{}{}^k - \bias_{\acc{}{}^k} - \noise_{\acc{}{}^k})dt
    \\
    \posi{}{}^{k+1} &= \posi{}{}^{k} + \vel{}{}^{k}dt + \frac{1}{2}\grav dt^2 
    + \frac{1}{2}\Rot{}{}^{k}(\accm{}{}^k - \bias_{\acc{}{}^k} - \noise_{\acc{}{}^k})dt^2
    \end{split}
    \label{eq:imu_dyn_disc}
\end{equation}
    
Now, these equations relate variables variables between two state variables at IMU frequency. To include these measurements in our smoothing estimator,
one solution would be to introduce new variables at the rate of the IMU. However, the size of the problem to solve would grow very quickly. A better option
is to integrate IMU measurements during extended periods of time. If we simply integrate in world frame by unfurling \ref{eq:imu_dyn_disc} between i and j, integrating
the sequence of IMU measurements $\mathcal{Z}_{ij}$, we obtain
the following equations:

\begin{equation}
    \begin{split}
    \Rot{}{}^{j}  &= \Rot{}{}^{i} \prod_{k=i}^{j-1} Exp((\angvelm{}{}^k - \bias_{\angvel{}{}^k} - \noise_{\angvel{}{}^k})dt) \\
    \vel{}{}^{j}  &= \vel{}{}^{i} + \sum_{k=i}^{j-1} \Big[\grav dt + \Rot{}{}^{k}(\accm{}{}^k - \bias_{\acc{}{}^k} - \noise_{\acc{}{}^k})dt \Big]  \\
    \posi{}{}^{j} &= \posi{}{}^{i} + \sum_{k=i}^{j-1} \Big[\vel{}{}^{k}dt + \frac{1}{2}\grav dt^2 
    + \frac{1}{2}\Rot{}{}^{k}(\accm{}{}^k - \bias_{\acc{}{}^k} - \noise_{\acc{}{}^k})dt^2 \Big]
    \end{split}
    \label{eq:IMUIntij}
\end{equation}

We could then directly derive a factor with an error function defined as
$e(\bfx_i, \bfx_j) = \bfDelta_{\bfx}(\bfx_i, \bfx_j)  \ominus  \bfDelta_{\mathcal{Z}_{ij}}(\bias_{\acc{}{}}^i, \bias_{\angvel{}{}}^j)$
assuming biases stay constant during $\Delta_{ij}$  where 

\begin{align}
    \bfDelta_{\bfx}(\bfx_i, \bfx_j) = 
    \begin{bsmallmatrix}
    \Rot{}{}^{i,T} \Rot{}{}^{j}  \\
    \vel{}{}^{j}  - \vel{}{}^{i}  \\
    \posi{}{}^{j} - \posi{}{}^{i}
    \end{bsmallmatrix}
\end{align}

and 

\begin{align}
    \bfDelta_{\mathcal{Z}_{ij}}(\bias_{\acc{}{}}^i, \bias_{\angvel{}{}}^j, \bfx_i) = 
    \begin{bsmallmatrix}
    \prod_{k=i}^{j-1} Exp((\angvelm{}{}^k - \bias_{\angvel{}{}^i})dt)  \\
    \sum_{k=i}^{j-1} \Big[\grav dt + \Rot{}{}^{k}(\accm{}{}^k - \bias_{\acc{}{}^k} - \noise_{\acc{}{}^k})dt \Big]  \\
    \sum_{k=i}^{j-1} \Big[\vel{}{}^{k}dt + \frac{1}{2}\grav dt^2 
    + \frac{1}{2}\Rot{}{}^{k}(\accm{}{}^k - \bias_{\acc{}{}^i})dt^2 \Big]
    \end{bsmallmatrix}
\end{align}

Both quantities are recomputed at each evaluation of the residual by the nonlinear solver. $\bfDelta_{\bfx}(\bfx_i, \bfx_j)$ is fairly quick to compute.
However $\bfDelta_{\mathcal{Z}_{ij}}(\bias_{\acc{}{}}^i, \bias_{\angvel{}{}}^j, \bfx_i)$ formulation requires to reintegrate the whole buffer of measurements which
is highly inefficient. This is due to the dependency on velocity, orientation and biases at timestamp $i$. However the terms of equation \ref{eq:IMUIntij} can 
be rearranged (proof in the annex -> TODO) differently to give:

\begin{equation}
    \begin{split}
    \bfDelta \Rot{}{ik} &\triangleq  \Rot{}{}^{i,T} \Rot{}{}^{j} =  \prod_{k=i}^{j-1} Exp((\angvelm{}{}^k - \bias_{\angvel{}{}^k} - \noise_{\angvel{}{}^k})dt) \\
    \bfDelta \vel{}{ik} &\triangleq \Rot{}{}^{i,T} (\vel{}{j} - \vel{}{i} - g \Delta t_{ij}) 
    = \prod_{k=i}^{j-1} \bfDelta \Rot{}{ik} Exp((\accm{}{}^k - \bias_{\acc{}{}^k} - \noise_{\acc{}{}^k})dt)  \\
    \bfDelta \posi{}{ik} &\triangleq \Rot{}{}^{i,T}(\posi{}{}^j - \posi{}{}^i - \vel{}{}^i \Delta t_{ij} - \frac{1}{2} \grav \Delta t_{ij}^2) 
    = \sum_{k=i}^{j-1} \Big[\bfDelta \vel{}{ik}dt +  \frac{1}{2} \bfDelta \Rot{}{ik} (\accm{}{}^k - \bias_{\acc{}{}^k} - \noise_{\acc{}{}^k})dt^2 \Big]
    \end{split}
    \label{eq:IMUPreintij}
\end{equation}

where we define $\bfDelta \Rot{}{ik} \triangleq \Rot{}{i,T} \Rot{}{k}$
We then overwrites our delta notations as $\bfDelta_{\bfx}(\bfx_i, \bfx_j) = \bfDelta_{\mathcal{Z}_{ij}}(\bias_{\acc{}{}}^i, \bias_{\angvel{}{}}^j)$
which lifts the dependency of the right hand term on the initial state $\bfx_i$. However the bias terms in \ref{eq:IMUPreintij} still enforce the repeated 
reintegration of the measurements buffer. Instead, the right hand side can be integrated once using biases estimates at the time where we begin to integrate them that
we note $\biasbar_i$. Then, we can approximate changes due to the reestimation of $\bias_i$ using a first order approximation 
$\bfDelta_{\mathcal{Z}_{ij}}(\bias^i) = \bar{\bfDelta}_{\mathcal{Z}_{ij}} \boxplus (\bfJ^{\bfDelta_{\mathcal{Z}_{ij}}}_{\bias_i} (\bias^i - \biasbar^i))$.

These two observations were first made by Lupton \cite{lupton-09}, whose formulation relied on Euler angles, and were later formalized on \SO(3) using Lie theory
by \cite{forster2017-TRO}. 

TOEXPLAIN:
- noise propa has to be done -> leads to very cumbersome formulas
- recursive formulation
- can be extended to other sensors -> generalized preintegration

\section{Generalized preintegration}
\section{IMU preintegration}
\section{IMU preintegration on Lie groups}
\section{External force preintegration}


\section{Related works}

\cite{hartley2018legged}
\cite{wisth2020preintegrated}
\cite{lupton-09}
\cite{forster2017-TRO}
GTSAM new IMU preint (?)
\cite{eckenhoff2019closed}
\cite{brossard2021associating}
\cite{luo2021unified} 