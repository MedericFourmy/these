\chapter{Kinematics}
\minitoc

The last measurement model of importance that we developed in this work blablabla
- First centroidal kinematics related to centroidal est and force preintegration
- Second Leg odometry


\section{Centroidal kinematics}
Use of the kinematic model of the robot to obtain measurements about centroidal quantities of the robot.

We need to relate base states to centroidal quantities to ground their estimate. 
For that, we can rely on the inertial-kinematic model to compute the CoM position wrt. base frame $\posi{B}{C}(\qa) \in \Reals^3$, the CoM velocity wrt. 
base frame $\vel{B}{C}(\qa, \dqa) \in \Reals^3$, the inertial matrix $\Inertia(\qa) \in \Reals^{3\times 3}$, and  the kinematic momentum due to gesticulation 
of the robot limbs $\AM_a(\qa, \dqa) \in \Reals^{3}$. 
As stated before, the computed CoM position is considered to be affected by Gaussian noise and a slowly varying bias $\posim{B}{C} = \posi{B}{C} + \bfb_{c} + \noise_c$. 
The angular velocity from the IMU is used and its bias has to be incorporated in the factor $\angvelm{}{} = \angvel{B}{B} + \bfb_{\omega} + \noise_{\omega}$.  
In the end the equations used to derive the factor are:
%
\begin{equation}
\small
\begin{split}
\COM &= \Rot{}{} (\posim{B}{C} -  \bias_{c} - \noise_{c}) + \posi{}{}
\\
\COMd &= 
\vel{}{} + \Rot{}{}((\angvelm{}{} - \bias_{\omega} - \noise_{\omega}) \times (\posim{B}{C} -  \bias_c - \noise_c) 
\\&~~~~~~+ (\velm{B}{C} - \noise_{v}))
\\
\AM &= \Rot{}{}(\Inertia (\angvelm{}{} - \bias_{\omega} - \noise_{\omega}) + \AM_a)
\end{split}
\end{equation}

Then, the residual $\bfr^{CK} \in \Reals^9$ is simply expressed as:
%
\begin{equation}
\small
\bfr^{CK}=
\begin{bmatrix}
\posim{B}{C} - (\Rot{}{}^T(\COM - \posi{}{}) + \bias_{c})
\\
(\angvelm{}{} - \bias_{\omega}) \times (\posim{B}{C} -  \bias_{p}) + \velm{B}{C} - \Rot{}{}^T(\COMd - \vel{}{})
\\
\Inertia (\angvelm{}{} - \bias_{\omega}) + \AM_a - \Rot{}{}^T\AM
\end{bmatrix}
\label{eq:CKFactor}
\end{equation}




\section{Leg odometry}
Assuming that we have access to an accurate contact detection, for each foot in contact between \keyframes $i$ and $j$, we can write that 
$\posi{W}{L}^i = \posi{W}{L}^j + \noise_{LO}$ where $\noise_{LO}$ denotes a Gaussian noise accounting for potential slip of the foot and kinematic inaccuracies. 
This pseudo measurement can be used to derive a 3D factor for each foot $l$ in contact:

\begin{equation}
    \bfr^{LO}_l(\posi{}{}^i, \Rot{}{}^i, \posi{}{}^j, \Rot{}{}^j) = \posi{}{}^i + \Rot{}{}^i \posim{B}{L}^i - (\posi{}{}^j + \Rot{}{}^j \posim{B}{L}^j)
\end{equation}

Where $\posim{B}{L}^i$ and $\posi{B}{L}^j$ are the contact positions in base frame at times $i$ and $j$ acquired from $\bfq_a$ via forward kinematics. 

Assuming good contact detection, leg odometry fused with the IMU is enough to properly observe 6 DoF motion with as few as 2 contact feet at a time. 
Note that contrary to previous works \cite{hartley2018legged, wisth2020preintegrated}, this leg odometry factor does not require gyroscope integration.

\section{Terrain height}

If we have some prior knowledge about the type of 

\section{Kinematic calibration}
A crucial part of the usability of this measurement model is to be able to compute an unbiased $\posi{B}{L}$ estimation for each foot.
