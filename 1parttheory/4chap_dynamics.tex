\chapter{Centroidal dynamics}
\minitoc


\section{Centroidal kinematics}
Use of the kinematic model of the robot to obtain measurements about centroidal quantities of the robot.

We need to relate base states to centroidal quantities to ground their estimate. 
For that, we can rely on the inertial-kinematic model to compute the CoM position wrt. base frame $\posi{B}{C}(\qa) \in \Reals^3$, the CoM velocity wrt. 
base frame $\vel{B}{C}(\qa, \dqa) \in \Reals^3$, the inertial matrix $\Inertia(\qa) \in \Reals^{3\times 3}$, and  the kinematic momentum due to gesticulation 
of the robot limbs $\AM_a(\qa, \dqa) \in \Reals^{3}$. 
As stated before, the computed CoM position is considered to be affected by Gaussian noise and a slowly varying bias $\posim{B}{C} = \posi{B}{C} + \bfb_{c} + \noise_c$. 
The angular velocity from the IMU is used and its bias has to be incorporated in the factor $\angvelm{}{} = \angvel{B}{B} + \bfb_{\omega} + \noise_{\omega}$.  
In the end the equations used to derive the factor are:
%
\begin{equation}
\small
\begin{split}
\COM &= \Rot{}{} (\posim{B}{C} -  \bias_{c} - \noise_{c}) + \posi{}{}
\\
\COMd &= 
\vel{}{} + \Rot{}{}((\angvelm{}{} - \bias_{\omega} - \noise_{\omega}) \times (\posim{B}{C} -  \bias_c - \noise_c) 
\\&~~~~~~+ (\velm{B}{C} - \noise_{v}))
\\
\AM &= \Rot{}{}(\Inertia (\angvelm{}{} - \bias_{\omega} - \noise_{\omega}) + \AM_a)
\end{split}
\end{equation}

Then, the residual $\bfr^{CK} \in \Reals^9$ is simply expressed as:
%
\begin{equation}
\small
\bfr^{CK}=
\begin{bmatrix}
\posim{B}{C} - (\Rot{}{}^T(\COM - \posi{}{}) + \bias_{c})
\\
(\angvelm{}{} - \bias_{\omega}) \times (\posim{B}{C} -  \bias_{p}) + \velm{B}{C} - \Rot{}{}^T(\COMd - \vel{}{})
\\
\Inertia (\angvelm{}{} - \bias_{\omega}) + \AM_a - \Rot{}{}^T\AM
\end{bmatrix}
\label{eq:CKFactor}
\end{equation}



\section{External force pre-integration}
In this section we apply the generalized pre-integration algorithm to the problem of using measured external
forces applied on a legged robot in a smoothing estimator. We propose to integrate the underactuated that we first recall. Then 
we derive the specificities of the pre-integration of external forces of a poly-articulated system. This is the main theoretical contribution
of our paper \cite{fourmy2021contact}.


\subsection{Centroidal dynamics}
\label{sec:centroidal_dynamics}
The robot dynamics is described by the well-known:
\begin{equation}\label{eq:wbdyn}
  \bfM(\q) \vq + h(\q,\vq) = \bm\tau_q + \sum_l \bfJ_l\tr \bff_l
\end{equation}
where $\q,\vq,\dvq,\bm\tau_q$ are the position, velocity, acceleration and torques of the robot in configuration space,
$\bff_l$ are the contact forces (written as 3D forces in this paper),
$\bfM$ is the generalized inertia matrix, $h$ the sum of gravity, Coriolis and centrifugal forces, and $\bfJ_l$ the jacobians of the contact points.
Because of the underactuated nature of legged robots, the configuration is often separated into $\q=(\posi{W}{B},\Rot{W}{B},\qa)$ where $\posi{W}{B}$ 
is the position in world frame of the robot base (typically, the torso or in our case the IMU), $\Rot{W}{B}$ the rotation of the base body with respect 
to the world and $\qa$ are the joint configuration of the actuated joints. 
%We will subsequently use the notation $\prescript{X}{}{[\cdot]}_Y$ to denote vectorial quantities of frame $Y$ expressed in frame $X$. A tilde $\Tilde{[\cdot]}$ denotes a noisy measurement.

While~\eqRef{eq:wbdyn} represents the whole dynamics, a subpart of it is of particular importance for legged robots.
The centroidal dynamics is written by the two equations:
%
\begin{equation}
    m \ddot{\bfc} = m \bfg + \sum_l \bff_l \quad , \quad
\dAM = \sum_l (\posi{}{l} - \COM) \times \bff_l
\label{eq:NewtonEuler}
\end{equation}
%
where $\COM,\AM$ are the position of the Center of Mass (CoM) and Angular Momentum (AM) around the CoM (both expressed in world frame), $m$ is the robot total mass, 
and the $\posi{}{l}$ are the positions of the contact points in world frame. The centroidal dynamics is an exact part of \eqRef{eq:wbdyn} and more deeply reveals 
the underactuation: the robot can move only if applying the proper forces to the environment, as the joint torques alone cannot lead to any modification 
of CoM or AM.


The classical approach in estimation of legged robot state is to first estimate the base state, and then to reconstruct the centroidal state $(\bfc,\dot{\bfc},\AM)$ using the joint position and velocity measurements, and the robot model.
This assumes that there is no direct measurement of the centroidal state.
Consequently, we are not be able to recover the exact centroidal state if there is any bias in the robot model.

Yet, we can see from the centroidal dynamics that the force measurements are connected to the variation of the centroidal state.
As observed in~\cite{carpentier2016center}, a proper fusion of the force measurements with an estimation of the state of the base makes the centroidal state observable.


\subsection{Force pre-integration factor}

The Newton-Euler equations \eqRef{eq:NewtonEuler} relate the evolution of the CoM and AM due to gravity, external forces and torques. 
In the case of a legged robot with punctual contact feet, only forces $\forcem{}{L}$ are applied at each limb contact $L$, with no torque. 
We assume that at each  limb contact we have access to noisy local force measurements $\forcem{}{L} = \force{L}{L} + \noise_f$. 
To transform them into the body frame $b$, we compute $\Rotm{}{L} \triangleq \Rot{B}{L}(\qa)  \in SO(3)$ and $\posim{}{L} \triangleq \posi{B}{L}(\qa) \in \Reals^3 $ from the joint configuration $\qa  \in \Reals^{12}$. 
The lever arm $(\posi{}{L}-\bias_c)$ in \eqRef{eq:NewtonEuler} uses a measurement of the CoM position in base frame $ \posi{B}{C}(\qa) \in \Reals^3$. 
This measure is biased due to inaccuracies in the robot model and therefore we add a bias variable to be estimated, $\bias_c \in \Reals^3$ so that $\posim{}{C} = \posi{B}{C}(\qa) + \bias_c + \noise_c$.
Assuming constant forces during each interval $\dt$ 
the integration of \eqRef{eq:NewtonEuler} yields the discrete dynamic model for the centroidal states:
%
\begin{equation}
\begin{split}
\COM^{k} &= \COM^{k-1} + \COMd^{k-1} \dt+\frac{1}{2} \bfg \dt^2 + \frac{1}{2m} \Rot{}{}^{k-1} \sum_L \Rotm{}{L}^k (\forcem{}{L}^k - \noise_{f}) \dt^2
\\
\COMd^{k}&= \COMd^{k-1} + \bfg \dt + \frac{1}{m} \Rot{}{}^k \sum_L\Rotm{}{L}^k (\forcem{}{L}^k - \noise_{f}) \dt 
\\
\AM^{k} &= \AM^{k-1} +\Rot{}{}^k \sum_L (\posim{}{L}^k  - \posim{}{C}^k +  \bias_{c}^k + \noise_{c}) \times \Rotm{}{L}^k(\forcem{}{L}^k - \noise_{f}) \dt
\end{split}
\label{eq:COMDiscrete}
\end{equation}
%
Analogously to the IMU case, it is possible to pre-integrate force measurements to derive a factor on the states 
$\bfx_c=[\COM, \COMd, \AM, \Rot{}{}]$ using deltas $\D_c=[\D\COM, \D\COMd, \D\AM, \DR]$. 
The rotation measured by the gyroscope has to be included too for the pre-integration to work. 
In this case, the bias vector is $\bias = [\bias_c, \bias_{\omega}]$. We define measurements  $\bfz^k$ to be:
%
\begin{equation}
\bfz^k = \left[ \posim{}{C}^k, \angvelm{}{}^k, \left[\forcem{}{L}^k, \posim{}{L}^k, \Rotm{}{L}^k \right]_{L=1..4}\right]
\end{equation}

The following operators are enough to particularize the  method in \secRef{sec:general-preint} to the contact forces case.
Integrating $\bfz_k$ during $\dt$ yields $\bm\delta_k$ in \eqRef{eq:data2delta} as
%
\begin{equation}
\small
    \bm\delta^{k}(\bfz^k, \bfb^i, \dt) =
    \begin{bmatrix}
    \frac{1}{2m} \sum_l \Rotm{}{L}^k \forcem{}{L}^k \dt^2
    \\
    \frac{1}{m} \sum_l \Rotm{}{L}^k \forcem{}{L}^k \dt 
    \\
    (\sum_l \posim{}{L}^k - (\posim{}{C}^k - \bias_c^i) \times (\Rotm{}{L}^k \forcem{}{L}^k ))\dt
    \\
    \Exp((\angvelm{}{}^k - \bias_{\omega}^i)\dt)
    \end{bmatrix}
\normalsize
\end{equation}
%
and the delta composition law in \eqRef{eq:deltaPlusDelta} as
%
\begin{equation}
    \D \circ \bm\delta = 
    \begin{bmatrix}
    \D\COM + \D\COMd \dt + \DR  \delta\COM
    \\
    \D\COMd + \DR  \delta \COMd
    \\
    \D\AM + \DR  \delta \AM
    \\
    \DR\dR
    \end{bmatrix}
    ~.
    \label{eq:DeltaDTCompo}
\end{equation}
%
The expected delta $\D^{ik}=\bfx^k\boxminus\bfx^i$ between KFs $i$ and $k$ reads
%
\begin{equation}
\D^{ik} =
\begin{bmatrix}
   \D\COM^{ik} \\ \D\COMd^{ik} \\ \D\AM^{ik} \\ \DR^{ik}
\end{bmatrix}
=
\begin{bmatrix}
	{\Rot{}{}^i}\tr (\COM^k - \COM^i - \COMd^i \Dt^{ik})
	\\
	{\Rot{}{}^i}\tr (\COMd^{k} - \COMd^{i} - \bfg \Dt^{ik})
	\\
	{\Rot{}{}^i}\tr (\AM^{k} - \AM^{i})
	\\
    {\Rot{}{}^i}\tr \Rot{}{}^k
\end{bmatrix}
~.
\end{equation}
%
Finally, the propagation of the state $\bfx_i$ to $\bfx_k$ using $\D_{ik}$, which can be used to retrieve a state at any time $k$ between KFs in the trajectory, is
%
\begin{equation}
	\bfx^k = \bfx^i \boxplus \D^{ik} =
	\begin{bmatrix}
	\COM^i + \COMd^i \Dt^{ik} + \Rot{}{}^i \Delta \COM^{ik} + \frac{1}{2} \bfg \Dt^{ik,2}
	\\
	\COMd^i + \Rot{}{}^i \Delta \COMd^{ik} + \grav \Dt^{ik}
	\\
	\AM^i + \Rot{}{}^i \Delta \AM^{ik}
	\\
	\Rot{}{}^i \DR^{ik}
	\end{bmatrix}
\end{equation}
%
%
%
%

