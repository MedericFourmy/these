\chapter{Underactuated dynamics and centroidal states}
\label{chp:underactuade_dynamics}
\minitoc
\bigskip


As mentioned in the literature review, centroidal states are key to the control of legged robots. Indeed, they provide rich information about the general
behavior of the system and can be used to check for the stability of the system. The centroidal state estimation is rich, especially for humanoid
robots (see \secRef{sec:centroidal_est_lit}). To our knowledge, all of the centroidal estimators proposed in the literature can be classified as loosely-coupled estimators 
(following the terminology introduced in \secRef{sec:proprio_filters}): they rely on prior knowledge of a base state. For instance, Piperakis \cite{piperakis2018nonlinear} describes the
whole pipeline: first, an inertial-kinematics EKF estimates the base state, then an EKF uses this fixed base state to obtain centroidal states. 
Since the factor graph optimization framework that we use theoretically reaches its full potential when the maximum number of cross-correlation
are considered, we did not find this solution satisfactory. In \cite{fourmy2021contact}, we proposed a tightly-coupled estimator that jointly estimates
base and centroidal states using IMU, kinematics, and force measurements.

We will first review the dynamical model of legged robots, then describe how we propose to use them as measurement models for quadruped and humanoid robots alike.
In particular, we design a new application of the generalized preintegration algorithm (see \secRef{sec:general-preint}) for force/torque measurements, which constitutes
one of the major theoretical contributions of this thesis.


\section{Centroidal dynamics}
\label{sec:centroidal_dynamics}
The robot dynamics is described by the Lagragian dynamics:
%
\begin{equation}\label{eq:wbdyn}
  \bfM(\q) \dvq + h(\q,\vq) = \bm\tau_q + \sum_l \bfJ_l\tr \wrench{}{l}
\end{equation}
%
where $\q,\vq,\dvq,\bm\tau_q$ are the position, velocity, acceleration and torques of the robot in configuration space,
$\wrench{}{l} \triangleq [\force{}{l},\bfm_l]$ are the contact wrenches (force and pure moment),
$\bfM$ is the generalized inertia matrix, $h$ the sum of gravity, Coriolis and centrifugal forces, and $\bfJ_l$ the jacobians of the contact points.
Because of the underactuated nature of legged robots, the configuration is often separated into $\q=(\posi{W}{B}, \prescript{W}{}{\bfq}_B ,\qa)$ where $\posi{W}{B}$ 
is the position of the robot base in world frame (typically, the torso or in our case the IMU), $\prescript{W}{}{\bfq}_B$ the orientation of the base body with respect 
to the world and $\qa$ are the joint configuration of the actuated joints. The configuration velocity has a similar separation, as described in \secRef{sec:forward_kinematics}.
%We will subsequently use the notation $\prescript{X}{}{[\cdot]}_Y$ to denote vectorial quantities of frame $Y$ expressed in frame $X$. A tilde $\Tilde{[\cdot]}$ denotes a noisy measurement.

While~\eqRef{eq:wbdyn} represents the whole dynamics, a subpart of it is of particular importance for legged robots.
The centroidal dynamics is written by the two equations:
%
\begin{equation}
    m \ddot{\bfc} = m \bfg + \sum_l \force{}{l} \quad , \quad
\dAM = \sum_l (\posi{}{l} - \COM) \times \force{}{l} + \torque{}{l}
\label{eq:NewtonEuler}
\end{equation}
%
where $\COM,\AM$ are the position of the Center of Mass (CoM) and Angular Momentum (AM) around the CoM (both expressed in world frame), $m$ is the robot total mass, 
and the $\posi{}{l}$ are the positions of the contact points in world frame. The centroidal dynamics is an exact part of \eqRef{eq:wbdyn} and more deeply reveals 
the underactuation: the robot can move only if applying the proper forces and torques to the environment, as the joint torques alone cannot lead to any modification 
of CoM or AM.

The classical approach in estimation of legged robot state is to first estimate the base state and then to reconstruct the centroidal state in world frame
%
\begin{equation}
    (\bfc,\dot{\bfc},\AM) \triangleq (\posi{W}{C},\vel{W}{C},\prescript{W}{}{\AM})
\end{equation}
%
using the joint position and velocity measurements, and the robot model. This assumes that there is no direct measurement of the centroidal state.
Consequently, we are not able to recover the exact centroidal state if there is any bias in the robot model.

Yet, we can see from the centroidal dynamics that the force measurements are connected to the variation of the centroidal state.
As observed in~\cite{carpentier2016center}, a proper fusion of the force measurements with an estimation of the state of the base makes the centroidal state observable.



\section{Centroidal kinematics}
%
We need to relate base states to centroidal quantities to ground their estimate. 
For that, we can rely on the inertial-kinematic model to compute the CoM position \wrt base frame $\posi{B}{C}(\qa) \in \Reals^3$, the CoM velocity \wrt 
base frame $\vel{B}{C}(\qa, \dqa) \in \Reals^3$, the inertial matrix $\Inertia(\qa) \in \Reals^{3\times 3}$, and the kinematic momentum due to gesticulation 
of the robot limbs $\AM_a(\qa, \dqa) \in \Reals^{3}$. The geometrical relation between base states and centroidal states is:
%
\begin{equation}
    \begin{split}
    \COM &= \Rot{}{} \posi{B}{C} + \posi{}{}
    \\
    \COMd &= 
    \vel{}{} + \Rot{}{}(\angvel{B}{B} \times \posim{B}{C} + \velm{B}{C})
    \\
    \AM &= \Rot{}{}(\Inertia \angvel{B}{B}+ \AM_a)
    \end{split}
\end{equation}


The computed CoM position is considered to be affected by Gaussian noise and a slowly varying bias 

\begin{equation}
    \posim{B}{C} = \posi{B}{C} + \bfb_{c} + \noise_c~.
    \label{eq:com_kine_meas}
\end{equation}

This slowly varying bias corresponds to the sum of the kinodynamic model inaccuracies whose dependence on the configuration $\qa$ is hard to model. Instead,
we will estimate this parameter by fusing centroidal kinematic measurements with the underactuated dynamics.

The angular velocity from the IMU is used and its bias has to be incorporated in the factor $\angvelm{}{} = \angvel{B}{B} + \bfb_{\omega} + \noise_{\omega}$.  
In the end, the equations used to derive the factor are:
%
\begin{equation}
    \begin{split}
    \COM &= \Rot{}{} (\posim{B}{C} -  \bias_{c} - \noise_{c}) + \posi{}{}
    \\
    \COMd &= 
    \vel{}{} + \Rot{}{} \left[ (\angvelm{}{} - \bias_{\omega} - \noise_{\omega}) \times (\posim{B}{C} -  \bias_c - \noise_c) + (\velm{B}{C} - \noise_{v}) \right]
    \\
    \AM &= \Rot{}{}(\Inertia (\angvelm{}{} - \bias_{\omega} - \noise_{\omega}) + \AM_a)
    \end{split}
\end{equation}

Then, the residual $\bfe^{CK} \in \Reals^9$ is then expressed as:
%
\begin{equation}
    \bfe^{CK}=
    \begin{bmatrix}
        \Rot{}{}\tr(\COM - \posi{}{}) + \bias_{c} - \posim{B}{C}
        \\
        \Rot{}{}\tr(\COMd - \vel{}{}) - \left[(\angvelm{}{} - \bias_{\omega}) \times (\posim{B}{C} -  \bias_{p}) + \velm{B}{C} \right]
        \\
        \Rot{}{}\tr\AM - \left[\Inertia (\angvelm{}{} - \bias_{\omega}) + \AM_a \right]
    \end{bmatrix}
    \label{eq:CKFactor}
\end{equation}

[Computation of Covariance in the annex?]



\section{External force pre-integration}
In this section, we apply the generalized pre-integration algorithm to the problem of using measured external
forces applied on a legged robot in a smoothing estimator. We propose to integrate the underactuated dynamics \eqRef{NewtonEuler} using wrench measurements.
To this end, we derive the specificities of the pre-integration of external forces of a poly-articulated system. This is the main theoretical contribution
of our paper \cite{fourmy2021contact}.

\subsection{Newton-Euler integration}
The Newton-Euler equations \eqRef{eq:NewtonEuler} relate the evolution of the CoM and AM due to gravity, external forces, and torques. 
In the case of a legged robot with punctual contact feet, only forces $\forcem{}{L}$ are applied at each limb contact $L$, with no torque. 
However, we will derive the equations in the general case where wrench are available (like for humanoids) since it is then easy to just set the 
torque terms to zero.

We assume that at each limb contact we have access to noisy local force $\forcem{}{L} = \force{L}{L} + \noise_f$ and pure torques 
$\torquem{}{L} = \torque{L}{L} + \noise_m$ measurements. 
To transform them into the body frame $b$, we compute $\Rotm{}{L} \triangleq \Rot{B}{L}(\qa)  \in SO(3)$ and $\posim{}{L} \triangleq \posi{B}{L}(\qa) \in \Reals^3 $ from the joint configuration $\qa  \in \Reals^{12}$ using forward kinematics \secRef{sec:forward_kinematics}. 
The lever arm $(\posi{}{L} - \bfc)$ in the Euler equation \eqRef{eq:NewtonEuler} uses a measurement of the CoM position in base frame $ \posi{B}{C}(\qa) \in \Reals^3$. 
This measure is biased and noisy as explained before: we use again the measurment model \eqRef{eq:com_kine_meas}.
Assuming constant forces during each interval $\dt$ 
the integration of \eqRef{eq:NewtonEuler} yields the discrete dynamic model for the centroidal states:
%
\begin{equation}
    \begin{split}
        \COM^{k} &= \COM^{k-1} + \COMd^{k-1} \dt+\frac{1}{2} \bfg \dt^2 + \frac{1}{2m} \Rot{}{}^{k-1} \sum_L \Rotm{}{L}^k (\forcem{}{L}^k - \noise_{f}) \dt^2
        \\
        \COMd^{k}&= \COMd^{k-1} + \bfg \dt + \frac{1}{m} \Rot{}{}^k \sum_L\Rotm{}{L}^k (\forcem{}{L}^k - \noise_{f}) \dt 
        \\
        \AM^{k} &= \AM^{k-1} +\Rot{}{}^k \sum_L \left[ (\posim{}{L}^k  - \posim{}{C}^k +  \bias_{c}^k + \noise_{c}) \times \Rotm{}{L}^k(\forcem{}{L}^k - \noise_{f}) 
                                                        + \Rotm{}{L}^k(\torquem{L}{L} - \noise_m)\right]\dt
    \end{split}
    \label{eq:COMDiscrete}
\end{equation}

Analogously to the IMU case, it is possible to pre-integrate force measurements to derive a factor on the states 
$\bfx_c=[\COM, \COMd, \AM, \Rot{}{}]$ by defining appropriate deltas $\D_c$ and $\boxminus$ verifying $\D^{im}=\bfx^k\boxminus\bfx^i$ between KFs $i$ and $m$. Developments similar to the IMU case (proof in \ref{sec:forster_proof}) lead us to define the expected deltas :
%
\begin{equation}
    \D^{im} =
    \begin{bmatrix}
    \D\COM^{im} \\ \D\COMd^{im} \\ \D\AM^{im} \\ \DR^{im}
    \end{bmatrix}
    =
    \begin{bmatrix}
        {\Rot{}{}^i}\tr (\COM^m - \COM^i - \COMd^i \Dt^{im})
        \\
        {\Rot{}{}^i}\tr (\COMd^{m} - \COMd^{i} - \bfg \Dt^{im})
        \\
        {\Rot{}{}^i}\tr (\AM^{m} - \AM^{i})
        \\
        {\Rot{}{}^i}\tr \Rot{}{}^m
    \end{bmatrix}~.
\end{equation}
%



The rotation measured by the gyroscope has to be included too for the pre-integration to work. 
In this case, the bias vector is $\bias = [\bias_c, \bias_{\omega}]$, that is, CoM and gyro biases. We define the measurements  $\bfz^k$ to be:
%
\begin{equation}
    \bfz^k = \left[ \posim{}{C}^k, \angvelm{}{}^k, \left[\forcem{}{L}^k, \torquem{L}{L}, \posim{}{L}^k, \Rotm{}{L}^k \right]_{L=1..4}\right]
\end{equation}

The following operators are enough to particularize the method in \secRef{sec:general-preint} to the contact forces/torque case.
Integrating $\bfz_k$ during $\dt$ yields $\bm\delta_k$ in \eqRef{eq:data2delta} as
%
\begin{equation}
        \bm\delta^{k}(\bfz^k, \bfb^i, \dt) =
        \begin{bmatrix}
        \frac{1}{2m} \sum_l \Rotm{}{L}^k \forcem{}{L}^k \dt^2
        \\
        \frac{1}{m} \sum_l \Rotm{}{L}^k \forcem{}{L}^k \dt 
        \\
        \sum_l\left[ (\posim{}{L}^k - (\posim{}{C}^k - \bias_c^i)) \times \Rotm{}{L}^k \forcem{}{L}^k + \Rotm{}{L}^k\torquem{}{L}^k) \right] \dt
        \\
        \Exp((\angvelm{}{}^k - \bias_{\omega}^i)\dt)
        \end{bmatrix}
    \normalsize
\end{equation}
%
and the delta composition law in \eqRef{eq:deltaPlusDelta} as
%
\begin{equation}
    \D \circ \bm\delta = 
    \begin{bmatrix}
    \D\COM + \D\COMd \dt + \DR  \delta\COM
    \\
    \D\COMd + \DR  \delta \COMd
    \\
    \D\AM + \DR  \delta \AM
    \\
    \DR\dR
    \end{bmatrix}
    ~.
    \label{eq:DeltaDTCompo}
\end{equation}
%
Finally, the propagation of the state $\bfx_i$ to $\bfx_k$ using $\D_{ik}$, which can be used to retrieve a state at any time $k$ between KFs in the trajectory, is
%
\begin{equation}
	\bfx^k = \bfx^i \boxplus \D^{ik} =
	\begin{bmatrix}
	\COM^i + \COMd^i \Dt^{ik} + \Rot{}{}^i \Delta \COM^{ik} + \frac{1}{2} \bfg \Dt^{ik,2}
	\\
	\COMd^i + \Rot{}{}^i \Delta \COMd^{ik} + \grav \Dt^{ik}
	\\
	\AM^i + \Rot{}{}^i \Delta \AM^{ik}
	\\
	\Rot{}{}^i \DR^{ik}
	\end{bmatrix}
\end{equation}


