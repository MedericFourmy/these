\chapter{Underactuated dynamics and centroidal states}
\label{chp:underactuade_dynamics}
\minitoc
\bigskip

In this chapter, we discuss how a tight coupling of the force measurements in the MAP formulation can lead to the direct estimation
of centroidal quantities, \ie\ the robot center of mass (CoM) position and velocity, and its angular momentum.


Centroidal estimation is crucial for legged robots balancing. However, this aspect has not been explored in the case of a factor graph 
estimator (whose application to legged-robots) is relatively recent. We propose to fuse the centroidal kinematics of the robots with the pre-integration of the force-torque measurements 
of the platform.

We will first review the dynamical model of legged robots, then describe how we propose to use them as measurement models for quadruped and humanoid robots alike.
In particular, we design a new application of the generalized pre-integration algorithm (see \secRef{sec:general-preint}) for force-torque measurements, which constitutes
one of the major theoretical contributions of this thesis.


\section{Centroidal dynamics}
\label{sec:centroidal_dynamics}
The robot dynamics is described by the Lagrangian dynamics:
%
\begin{equation}\label{eq:wbdyn}
  \bfM(\q) \dvq + h(\q,\vq) = \bm\tau_q + \sum_l \bfJ_l\tr \wrench{}{l}
\end{equation}
%
where $\q,\vq,\dvq,\bm\tau_q$ are the position, velocity, acceleration and torques of the robot in configuration space,
$\wrench{}{l} \triangleq [\force{}{l},\bfm_l]$ are the contact force-torques,
$\bfM$ is the generalized inertia matrix, $h$ the sum of gravity, Coriolis and centrifugal forces, and $\bfJ_l$ the Jacobians of the contact points.
Because of the underactuated nature of legged robots, the configuration is often separated into $\q=(\posi{W}{B}, \prescript{W}{}{\bfq}_B ,\qa)$ where $\posi{W}{B}$ 
is the position of the robot base in world frame (typically, the torso or in our case the IMU), $\prescript{W}{}{\bfq}_B$ the orientation of the base body with respect 
to the world and $\qa$ are the joint configuration of the actuated joints. The configuration velocity has a similar separation, as described in \secRef{sec:forward_kinematics}.
%We will subsequently use the notation $\prescript{X}{}{[\cdot]}_Y$ to denote vectorial quantities of frame $Y$ expressed in frame $X$. A tilde $\Tilde{[\cdot]}$ denotes a noisy measurement.

While~\eqRef{eq:wbdyn} represents the whole dynamics, a subpart of it is of particular importance for legged robots.
The centroidal dynamics is written by the two equations:
%
\begin{equation}
    m \ddot{\bfc} = m \bfg + \sum_l \force{}{l} \quad , \quad
\dAM = \sum_l (\posi{}{l} - \COM) \times \force{}{l} + \torque{}{l}
\label{eq:NewtonEuler}
\end{equation}
%
where $\COM,\AM$ are the position of the Center of Mass (CoM) and Angular Momentum (AM) around the CoM (both expressed in world frame), $m$ is the robot total mass, 
and the $\posi{}{l}$ are the positions of the contact points in world frame. The centroidal dynamics is an exact part of \eqRef{eq:wbdyn} and more deeply reveals 
the underactuation: the robot can move only if applying the proper forces and torques to the environment, as the joint torques alone cannot lead to any modification 
of CoM or AM.

The classical approach in legged robot state estimation is to first estimate the base state and then to reconstruct the centroidal state in world frame
%
\begin{equation}
    (\bfc,\dot{\bfc},\AM) \triangleq (\posi{W}{C},\vel{W}{C},\prescript{W}{}{\AM})
\end{equation}
%
using the joint position and velocity measurements, and the robot model. This assumes that there is no direct measurement of the centroidal state.
Consequently, we are not able to recover the exact centroidal state if there is any bias in the robot model.

Yet, we can see from the centroidal dynamics that the force measurements are connected to the variation of the centroidal state.
As observed in~\cite{carpentier2016center}, a proper fusion of the force measurements with an estimation of the state of the base makes the centroidal state observable.
We achieve this by adding two different kinds of information to the factor graph: centroidal kinematics and pre-integrated forces and torques. The derivation of their involved factors is presented hereafter.


\section{Centroidal kinematics}
\label{sec:centroidal_kinematics}
%
We need to relate base states to centroidal quantities to ground their estimate. 
For that, we can rely on the inertial-kinematic model to compute the CoM position \wrt base frame $\posi{B}{C}(\qa) \in \Reals^3$, the CoM velocity \wrt 
base frame $\vel{B}{C}(\qa, \dqa) \in \Reals^3$, the inertial matrix $\Inertia(\qa) \in \Reals^{3\times 3}$, and the kinematic momentum due to gesticulation 
of the robot limbs $\AM_a(\qa, \dqa) \in \Reals^{3}$. The geometrical relation between base states and centroidal states is:
%
\begin{equation}
    \begin{split}
    \COM &= \Rot{}{} \posi{B}{C} + \posi{}{}
    \\
    \COMd &= 
    \vel{}{} + \Rot{}{}(\angvel{B}{B} \times \posim{B}{C} + \velm{B}{C})
    \\
    \AM &= \Rot{}{}(\Inertia \angvel{B}{B}+ \AM_a)
    \end{split}
\end{equation}

By definition, computation of the CoM position from the kinodynamic model resorts to the formula

\begin{equation}
    \posi{B}{\COM} = \sum_{i=1}^{n} m_i \posi{B}{\COM_i}(\qa),
\end{equation}
%
were $m_i$ are the individual segment masses and $\posi{B}{\COM_i}$ the levers between the base and the center of mass of each segment, computed from the robot configurations $\qa$. In general, these terms are not exactly accurate as our model of the robot may not be faithful to the real system. The computation of $\posi{B}{\COM}$ is therefore biased. Worse, this bias is not constant, since it nonlinearly depends on the robot limbs configuration. 

We chose a simple additive model on the CoM kinematics noisy measurements

\begin{equation}
    \posim{B}{C} = \posi{B}{C} + \bfb_{c} + \noise_c~,
    \label{eq:com_kine_meas}
\end{equation}
%
where $\noise_c$ is a Gaussian noise and $\bfb_{c}$ is a varying bias that we wish to estimate.


The angular velocity from the IMU is used and its bias $\bfb_\bw$ has to be incorporated in the factor, $\angvelm{}{} = \angvel{B}{B} + \bfb_{\omega} + \noise_{\omega}$.  
In the end, the equations used to derive the factor are:
%
\begin{equation}
    \begin{split}
    \COM &= \Rot{}{} (\posim{B}{C} -  \bias_{c} - \noise_{c}) + \posi{}{}
    \\
    \COMd &= 
    \vel{}{} + \Rot{}{} \left[ (\angvelm{}{} - \bias_{\omega} - \noise_{\omega}) \times (\posim{B}{C} -  \bias_c - \noise_c) + (\velm{B}{C} - \noise_{v}) \right]
    \\
    \AM &= \Rot{}{}(\Inertia (\angvelm{}{} - \bias_{\omega} - \noise_{\omega}) + \AM_a)
    \end{split}
\end{equation}

The residual $\bfe^{CK} \in \Reals^9$ is finally expressed as:
%
\begin{equation}
    \bfe^{CK}(\posi{}{}, \vel{}{}, \Rot{}{}, \COM, \COMd, \AM, \bias_c, \bias_{\angvel{}{}})=
    \begin{bmatrix}
        \Rot{}{}\tr(\COM - \posi{}{}) + \bias_{c} - \posim{B}{C}
        \\
        \Rot{}{}\tr(\COMd - \vel{}{}) - \left[(\angvelm{}{} - \bias_{\omega}) \times (\posim{B}{C} -  \bias_{p}) + \velm{B}{C} \right]
        \\
        \Rot{}{}\tr\AM - \left[\Inertia (\angvelm{}{} - \bias_{\omega}) + \AM_a \right]
    \end{bmatrix}
    \label{eq:CK_residual}
\end{equation}

This factor is added to the factor graph as a link between base states and centroidal states and, therefore, depends on almost all variables of the problem
at a certain timestamp. We use this factor in the application presented in \chpRef{chp:centroidal_estimation}, whose factor graph is represented in 
\figRef{fig:centroidal_factor_graph}.



\section{Force-torque pre-integration}
\label{sec:force_torque_preint}
In this section, we apply the generalized pre-integration algorithm to the problem of using measured external
forces applied on a legged robot in a smoothing estimator. We propose to integrate the underactuated dynamics \eqRef{eq:NewtonEuler} using force-torque measurements.
To this end, we derive the specificities of the pre-integration of external forces of a poly-articulated system. This is the main theoretical contribution
of this chapter.

\subsection{Newton-Euler integration}
The Newton-Euler equations \eqRef{eq:NewtonEuler} relate the evolution of the CoM and AM due to gravity, external forces, and torques. 
In the case of a legged robot with punctual contact feet, only forces $\forcem{}{L}$ are applied at each limb contact $L$, with no torque. 
However, we will derive the equations in the general case where force-torque are available (like for humanoids) since it is then easy to just set the 
torque terms to zero.

We assume that at each limb contact we have access to noisy local force $\forcem{}{L} = \force{L}{L} + \noise_f$ and pure torques 
$\torquem{}{L} = \torque{L}{L} + \noise_m$ measurements. 
To transform them into the body frame $b$, we compute $\Rotm{}{L} \triangleq \Rot{B}{L}(\qa)  \in SO(3)$ and $\posim{}{L} \triangleq \posi{B}{L}(\qa) \in \Reals^3 $ 
from the joint configuration $\qa  \in \Reals^{12}$ using forward kinematics \secRef{sec:forward_kinematics}. 
The lever arm $(\posi{}{L} - \bfc)$ in the Euler equation \eqRef{eq:NewtonEuler} uses a measurement of the CoM position in base frame $ \posi{B}{C}(\qa) \in \Reals^3$. 
This measure is biased and noisy as explained before: we use again the measurement model \eqRef{eq:com_kine_meas}.
Assuming constant forces during each interval $\dt$ the integration of \eqRef{eq:NewtonEuler} yields the discrete dynamic model for the centroidal states:


\begin{equation}
    \begin{split}
        \COM^{k} &= \COM^{k-1} + \COMd^{k-1} \dt+\frac{1}{2} \bfg \dt^2 + \frac{1}{2m} \Rot{}{}^{k-1} \sum_L \Rotm{}{L}^k (\forcem{}{L}^k - \noise_{f}) \dt^2
        \\
        \COMd^{k}&= \COMd^{k-1} + \bfg \dt + \frac{1}{m} \Rot{}{}^k \sum_L\Rotm{}{L}^k (\forcem{}{L}^k - \noise_{f}) \dt 
        \\
        \AM^{k} &= \AM^{k-1} +\Rot{}{}^k \sum_L \left[ (\posim{}{L}^k  - \posim{}{C}^k +  \bias_{c}^k + \noise_{c}) \times \Rotm{}{L}^k(\forcem{}{L}^k - \noise_{f}) 
                                                        + \Rotm{}{L}^k(\torquem{L}{L} - \noise_m)\right]\dt
        \\
        \Rot{}{}^k &= \Rot{}{}^{k-1}\Exp(\angvelm{}{}^k - \bias_{\angvel{}{}}^k - \noise_{\angvel{}{}}^k)
    \end{split}
    \label{eq:COMDiscrete}
\end{equation}

This model is well adapted to humanoid robots with 6axis force-torque sensors at their flat feet or quadrupeds with 3D force sensors at their (approximately) 
point feet. Indeed, for quadruped with point feet, no pure moment is applied by the ground of the robot. In this case, the same equations can be kept, simply setting the
pure torques  $\torque{}{}$ terms to zero.

Note that the orientation of the system $\Rot{}{}$ appears in the centroidal states equations to relate the local kinematic and force measurement to the world frame in
which we conduct the integration. We, therefore, have to include the dynamical model of the orientation by integrating gyroscope measurements $\angvelm{}{}$ together with forces and torques. 

As illustrated in \secRef{sec:imu_preint_motivation}, recursively applying \eqRef{eq:COMDiscrete} and defining a naive difference in the world frame leads to an inefficient
residual definition. Instead, we resort to the generalized pre-integration algorithm that we developed in \ref{sec:general-preint}.






\subsection{Force-torque pre-integration on composite Lie group}

This section follows the same structure and logic as the two IMU pre-integration algorithm described in \ref{sec:imu_preint_composite} and \ref{sec:imu_preint_compact} since
we use the same generalized pre-integration pipeline. 
% To apply this pipeline to a new motion sensor, we need to define the delta $\D_c$ Lie group relating the centroidal states (and base orientation) between two \keyframes\ thanks to the $\boxminus$ operator 
% $\D^{im}=\bfx^k\boxminus\bfx^i$. 
The sensor measurements $\bfz^k$ involved in the integration are kinematics, gyro, and force-torques  measurements:
%
\begin{equation}
    \bfz^k = \left[ \posim{}{C}, \angvelm{}{}, \left[\forcem{}{L}, \torquem{L}{L}, \posim{}{L}, \Rotm{}{L} \right]_{L=1..4}\right]^k
\end{equation}
%
that we assume to be synchronized.

As we explained, $\posim{}{C}$ and $\angvelm{}{}$ are corrupted by noise and biases. Thus, these biases  $\bias = [\bias_c, \bias_{\omega}]$ are added to the estimator.
As we did in the IMU case, we assume them constant between two \keyframes\ and affected by a random walk drift.


\subsubsection{Definition of the delta group}

The state variables on which the pre-integration is applied are defined as $\bfx=[\COM, \COMd, \AM, \Rot{}{}]$.
Note that the rotation has to be included to be able to define residuals in the local frame.
% By recursively applying \eqRef{eq:COMDiscrete} between timestamps $i$ and $m$ and rearranging the terms \footnote{proof similar to the IMU case described in 
% \secRef{sec:forster_proof}}, we can write an equation of the sort
% %
% \begin{equation}
%     \widehat\D_{im}(\bfx_i, \bfx_m) = \D_{im}(\bias_i),
% \end{equation}
% %
% where the expected motion delta $\widehat\D^{im} = [\widehat\D\COM^{im}, \widehat\D\COMd^{im}, \widehat\D\AM^{im}, \widehat\DR^{im}]$ are related to the states through the 
% $\boxminus$ operator by

The definition of delta and $\boxminus$ operator comprises two parts. For the  force-torque pre-integration onto centroidal dynamics, we observe that a body subject to no force and torque will free-fall at the acceleration of gravity while keeping a constant angular momentum. For the orientation part, as we did for the IMU, a null angular velocity means a non-rotating frame. The deltas, defined as the motion between such free-falling frames and the current state, can therefore be written as:
%
\begin{equation}
    \widehat\D^{im} = \bfx^m \boxminus \bfx^i 
    \triangleq
    \begin{bmatrix}
        \widehat\D\COM^{im} \\ \widehat\D\COMd^{im} \\ \widehat\D\AM^{im} \\ \widehat\DR^{im}
    \end{bmatrix}
    =
    \begin{bmatrix}
        {\Rot{}{}^i}\tr (\COM^m - \COM^i - \COMd^i \Dt_{im} - \frac{1}{2} \bfg \Dt_{im}^{2})
        \\
        {\Rot{}{}^i}\tr (\COMd^{m} - \COMd^{i} - \bfg \Dt_{im})
        \\
        {\Rot{}{}^i}\tr (\AM^{m} - \AM^{i})
        \\
        {\Rot{}{}^i}\tr \Rot{}{}^m
    \end{bmatrix}~.
    \label{eq:com_boxminus}
\end{equation}
%
With this definition, the pre-integrated deltas are guaranteed to depend only on the sensor measurements and bias, and not on the initial state $\bfx^i$. The inverse operation $\boxplus$ is obtained as
%
\begin{equation}
    \bfx^m = \bfx^i \boxplus \D^{im} \triangleq
    \begin{bmatrix}
    \COM^i + \COMd^i \Dt^{im} + \Rot{}{}^i \Delta \COM^{im} + \frac{1}{2} \bfg \Dt_{im}^{2}
    \\
    \COMd^i + \Rot{}{}^i \Delta \COMd^{im} + \grav \Dt^{im}
    \\
    \AM^i + \Rot{}{}^i \Delta \AM^{im}
    \\
    \Rot{}{}^i \DR^{im}
    \end{bmatrix}
\end{equation}


\subsubsection{Definition of the group operations}

Given two centroidal deltas $\D=[\D\COM,\D\COMd,\D\AM, \DR]$ and $\bfdelta=[\delta\COM,\delta\COMd,\delta\AM, \dR]$, 
the group composition law $\D=\D\circ\bm\delta$ is defined as
%
\begin{equation} 
    \D \circ \bm\delta
    =
    \begin{bmatrix}
        \D\COM + \D\COMd\dt + \DR\delta\COM \\
        \D\COMd + \DR\delta\COMd \\
        \D\AM + \DR\delta\AM \\
        \DR\dR 
    \end{bmatrix}
    \label{eq:composition_delta_com}
\end{equation}
%
with a group identity element composed of the identity element of its components:
%
\begin{equation}
    \D_{\cE} = \begin{bmatrix}
    \bf0_3\\ \bf0_3 \\ \bf0_3 \\ \bfI_3
    \end{bmatrix},
\end{equation}
%
the resulting inverse being
%
\begin{equation}
    \D^{-1} =     
    \begin{bmatrix}
    - \DR\tr(\D\COM + \D\COMd \Dt) \\
    - \DR\tr \D\COMd \\
    - \DR\tr \D\AM \\
      \DR\tr
    \end{bmatrix}.
\label{eq:inverse_delta_com}
\end{equation}
%
The calibration function $c()$ and exponential map $\Exp()$ are combined to integrate one step of measurement $\tilde\bfz$,
yielding $\bm\delta=f(\tilde\bfz,\bfb,\dt)$ in \eqRef{eq:data2delta_com} as
%
\begin{equation}
    \bm\delta^{k}(\bfz^k, \bfb^i, \dt) =
    \begin{bmatrix}
    \frac{1}{2m} \sum_l \Rotm{}{L}^k \forcem{}{L}^k \dt^2
    \\
    \frac{1}{m} \sum_l \Rotm{}{L}^k \forcem{}{L}^k \dt 
    \\
    \sum_l\left[ (\posim{}{L}^k - (\posim{}{C}^k - \bias_c^i)) \times \Rotm{}{L}^k \forcem{}{L}^k + \Rotm{}{L}^k\torquem{}{L}^k) \right] \dt
    \\
    \Exp((\angvelm{}{}^k - \bias_{\omega}^i)\dt)
    \end{bmatrix}
    \label{eq:data2delta_com}
\end{equation}



\subsubsection{Delta pre-integration and factor residual}

The concatenation of the operations above lead exactly the algorithm we implemented in \cite{fourmy2021contact}:
%
\begin{itemize}
    \item Initialize $\D_{ii} = \D_\cE$, $\Sigma_\D=\bf0$, $\bfJ^\D_\bfb = \bf0$, and $\ol\bfb_i = \bfb_i$.
    \item Calibrate data and retract to manifold using \eqRef{eq:data2delta_com}.
    \item Compose $\ol\D$ using \eqRef{eq:composition_delta_com}.
\end{itemize}
%
The factor residual is again computed following \secRef{sec:preint_residual} exactly.



\subsubsection{Comment on the link with IMU composite pre-integration}
The reader will have noticed the high similarity between the centroidal motion delta group defined here and the composite IMU delta Lie group defined 
in \secRef{sec:imu_preint_composite} (\eg between equations \eqRef{eq:com_boxminus} and \eqRef{eq:boxminus-forster}). If we only consider the center of mass, 
its velocity, and the orientation of the base equations, they are in fact the same. 
Indeed, the force sensors provide local second-order information on the CoM position, which is akin to the IMU acceleration measurements. 

Where the problems differ however is in the inclusion of the angular momentum (the second part of the Newton-Euler equations \eqRef{eq:NewtonEuler}). This equation introduces the dependence on 
the CoM lever bias $\bias_c$, which is also present in the centroidal kinematics residual \eqref{eq:CK_residual}. \cite{rotella2015humanoid} showed that including the same set of measurements on centroidal quantities as
us, the state space centroidal system with centroidal kinematics bias is made observable, provide we have a prior estimation of the base states. In our case, base states are
state variables estimated in conjunction with the centroidal states. However, if the base states are made observable by other sensor sources, then the same observability analysis
should apply to the centroidal states in our case. This is confirmed by experimental results presented in \ref{chp:centroidal_estimation}.


\section{Conclusion}

In this chapter, we have proposed the first step toward a whole-body estimator
based on MAP. On the first hand, we have introduced centroidal states into the MAP
problem, in addition to the previous decision variables, classical estimated in SLAM.
The centroidal states are related to the basis states through the centroidal kinematics,
which is typically used in state-of-the-art estimators. While the same relation is
here used, the classical centroidal estimator only loosely couples these two estimated
quantities, then being unable to exploit other observations on the centroidal states
to also contribute to the estimation of the basis.
Then we draw the relation between the force measurements and the centroidal states,
which enriches the MAP factor graph and enables us to take advantage of the tight coupling
between basis and centroidal states. The force sensors are related to the centroidal states
in a somehow similar relation to the IMU is connected to the base state. As shown
previously on more theoretical estimation work, the force sensors provide the
observability condition to unbias the COM position despite (unavoidable) kinematic
and inertial calibration problem of the robot model.

We will show in the experimental part of this thesis how this tight coupling can
be implemented on a quadruped robot, although a significant experimental work
remains to be done to demonstrate its interest in practice. As previously said, this
contribution is a first step toward building a whole-body estimator, \ie an estimator
that would be able to estimate all robot related quantities by fusing any
available sensors, in particular, grid sensors such as robot skin and distributed IMUs.
