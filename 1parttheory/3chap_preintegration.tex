\chapter{High-rate motion data pre-integration}
\label{chp:preintegration}
\minitoc
\bigskip


Sensor modalities available on robotic platforms display a great variety in the rate at which measurements are acquired. For instance, 
a camera may record images at 33Hz while an IMU may be updated at 1kHz. In the context of Factor Graph estimation, this creates a challenge since residuals
are defined between \keyframes\ selected at a relatively low frequency, at times as low as 1Hz. One needs to integrate measurements from these high-rate sensors between \keyframes\,
which is not trivial in general.

In this chapter, we first describe the example of the integration of IMU measurements which motivates the development of the pre-integration theory \cite{lupton-09,forster2017-TRO}.
We then express this pre-integration in more abstract mathematical terms so as to generalize it to other high-rate sensors \cite{atchuthan-18-thesis}. 
Then, we describe a possible reformulation of the original IMU-preintegration algorithm \cite{forster2017-TRO} by exploiting a new Lie group that we 
proposed in \cite{fourmy2019absolute}. We conclude the chapter with a discussion of related works.

  
\section{A motivational example: IMU integration for graph optimization}
\label{sec:imu_preint_motivation}

In this section we will introduce the IMU measurement model and explain why a naive integration of these measurements in the world frame does not immediately
lead to a viable factor. We will then introduce the observations that lead to the development of the so-called IMU pre-integration algorithm by Lupton \cite{lupton-09}.

Let us consider the estimation of a robot base state comprised of its pose and velocity in the world frame,
%
\begin{equation}
    \bfx = [\posi{W}{WB}, \vel{W}{WB}, \Rot{W}{B}]
    \triangleq 
    [\bfp, \bfv, \Rot{}{}].
\end{equation}

We make a few hypothesis. First, we neglect effects due the rotation of the Earth by assuming 
that our world frame (which is fixed \wrt the ground) is an inertial frame. This is a common simplification in robotics scenarios \cite{forster2017-TRO}. 
Second, without loss of generality, we assume that the IMU frame is identical to the base frame in the following developments.

The IMU measurements are known to be noisy, biased, and affected by the gravity,
%
\begin{equation}
    \begin{split}
    \angvelm{}{} &= \angvel{B}{WB} + \bias_{\angvel{}{}} + \noise_{\angvel{}{}} 
    \\
    \accm{}{}    &= \acc{B}{WB} + \bias_{\acc{}{}} + \noise_{\acc{}{}} + \Rot{B}{W} \grav .
    \end{split}
    \label{eq:imu_meas_model}
\end{equation}
%    
Other sensors can help estimate the IMU biases $\bias \triangleq [\bias_{\acc{}{}}, \bias_{\angvel{}{}}]$ that are thus included in the estimator state.


Once the IMU has been started, these biases may change over time, more or less slowly depending on the quality of the IMU. This change is modeled 
as a random walk, which is close to the observed behavior for reasonably short periods of time \cite{hussen2015low}.
A dynamical model continuous-time based on strapdown integration of IMU measurements can then be derived:
%
\begin{align}
\begin{split}
    \dot{\posi{}{}} &= \vel{}{} \\
    \dot{\vel{}{}} &= \acc{W}{WB} \\
    \dot{\Rot{}{}} &= \Rot{}{} [\angvel{B}{WB}]_{\times} \\
    \dot{\bias}_{\bfa} &= \bfw_{\bfa}^c \\
    \dot{\bias}_{\angvel{}{}} &= \bfw_{\angvel{}{}}^c ,
\end{split}
\label{eq:imu_dyn_conti}
\end{align}
%
where $\bfw_{\bfa}^c \in \mathcal{N}(0,\bfW_\bfa^c)$ and $\bfw_{\bw}^c \in \mathcal{N}(0,\bfW_\bw^c)$ are the bias' random walk continuous-time white noises.

Introducing the measurement equations \eqRef{eq:imu_meas_model} in the continuous dynamics equations \eqRef{eq:imu_dyn_conti} and using a zero order hold
explicit Euler integration scheme during $\dt$ results in the discrete dynamics:
%
\begin{equation}
    \begin{split}
    \posi{}{}^{k+1} &= \posi{}{}^{k} + \vel{}{}^{k}\dt + \frac{1}{2}\grav \dt^2 
    + \frac{1}{2}\Rot{}{}^{k}(\accm{}{}^k - \bias_{\acc{}{}}^k - \noise_{\acc{}{}}^k)\dt^2 \\
    \vel{}{}^{k+1}  &= \vel{}{}^{k} + \grav \dt + \Rot{}{}^{k}(\accm{}{}^k - \bias_{\acc{}{}}^k - \noise_{\acc{}{}}^k)\dt
    \\
    \Rot{}{}^{k+1}  &= \Rot{}{}^{k}\Exp((\angvelm{}{}^k - \bias_{\angvel{}{}}^k - \noise_{\angvel{}{}}^k)\dt)
    \\
    \bfb_\bfa^{k+1} &= \bfb_\bfa^k + \bfw_\bfa\\
    \bfb_{\angvel{}{}}^{k+1} &= \bfb_{\angvel{}{}}^k + \bfw_{\angvel{}{}}
    \end{split}
    \label{eq:imu_dyn_disc}
\end{equation}
%
where $\bfw_{\bfa} \in \mathcal{N}(0,\bfW_\bfa)$ and $\bfw_{\bw} \in \mathcal{N}(0,\bfW_\bw)$ are the bias' random walk discrete-time white noises, satisfying $\bfW_\bfa=\bfW_\bfa^c \dt$ and $\bfW_\bw=\bfW_\bw^c\dt$.

    
Now, these equations relate consecutive states with data sampled at IMU frequency. To include these measurements in our smoothing estimator,
one solution would be to introduce new states at the rate of the IMU. However, the size of the problem would grow very quickly. A better option
is to integrate IMU measurements during extended periods of time. If we simply integrate the sequence of IMU measurements $\cZ_{im}$ between timestamps 
$t_i$ and $t_m$ by recursively applying \eqRef{eq:imu_dyn_disc}, we obtain:
%
\begin{equation}
    \begin{split}
    \posi{}{}^{m} &= \posi{}{}^{i} + \sum_{k=i}^{m} \Big[\vel{}{}^{k}\dt + \frac{1}{2}\grav \dt^2 
    + \frac{1}{2}\Rot{}{}^{k}(\accm{}{}^k - \bias_{\acc{}{}}^k - \noise_{\acc{}{}}^k)\dt^2 \Big] \\
    \vel{}{}^{m}  &= \vel{}{}^{i} + \sum_{k=i}^{m} \Big[\grav \dt + \Rot{}{}^{k}(\accm{}{}^k - \bias_{\acc{}{}}^k - \noise_{\acc{}{}}^k)\dt \Big]  \\
    \Rot{}{}^{m}  &= \Rot{}{}^{i} \prod_{k=i}^{m} \Exp((\angvelm{}{}^k - \bias_{\angvel{}{}}^k - \noise_{\angvel{}{}}^k)\dt) 
    \end{split}
    \label{eq:IMUIntij}
\end{equation}
%
where we assume that IMU biases stay constant during $\Dt^{im}$
\begin{equation*}
    \bias^k \approx \bias^i  ~~~ \forall k \in [i..m],
\end{equation*}
%
which makes their integration trivial and thus allows us to concentrate in the upper three lines of the model.




We are here in the position of illustrating why a naive definition of the data integration leads to very bad performance. 
By observing \eqRef{eq:IMUIntij} we can define a motion error, as follows. 
First, we naively define the motion increments or "deltas"
%
% \begin{equation}
%     \D_{im} = \left[\Dp_{im}, \Dv_{im}, \DR_{im} \right] \quad , \quad (\Dp_{im},\Dv_{im},\DR_{im}) \in \cM_{\D}=\left< \Reals^3,\Reals^3, \SO(3) \right>
%     \label{eq:delta_imu_def}
% \end{equation}
%
in two ways. 
%Note that delta quantities here belong to the composite Lie group $\left< \Reals^3,\Reals^3\in \SO(3) \right>$. 
From the integration of the motion model, %we get
%
\begin{align}
    \D_{im}(\bias^i, \bfx^i) = 
    \begin{bmatrix}
    \Dp_{im}\\ \Dv_{im}\\ \DR_{im}
    \end{bmatrix} \triangleq
    \begin{bmatrix}
    \sum_{k=i}^{m} \Big[\vel{}{}^{k}\dt + \frac{1}{2}\grav \dt^2 
    + \frac{1}{2}\Rot{}{}^{k}(\accm{}{}^k - \bias_{\acc{}{}}^i - \noise_{\acc{}{}}^k)\dt^2 \Big] \\
    \sum_{k=i}^{m} \Big[\grav \dt + \Rot{}{}^{k}(\accm{}{}^k - \bias_{\acc{}{}}^i - \noise_{\acc{}{}}^k)\dt \Big]  \\
    \prod_{k=i}^{m} \Exp((\angvelm{}{}^k - \bias_{\angvel{}{}}^i- \noise_{\angvel{}{}}^k)\dt)  
    \end{bmatrix}
    \label{eq:imu_delta_naive}
\end{align}
%
and from the difference between initial and final states:
%
\begin{align}
    \hat\D_{im}(\bfx^i, \bfx^m) = 
    \begin{bmatrix}
    \hat\Dp_{im}\\ \hat\Dv_{im}\\ \hat\DR_{im}
    \end{bmatrix} \triangleq
    \begin{bmatrix}
    \posi{}{}^{m} - \posi{}{}^{i} \\
    \vel{}{}^{m}  - \vel{}{}^{i}  \\
    \Rot{}{}^{i,T} \Rot{}{}^{m}  
    \end{bmatrix}.
\end{align}
%
Then, we build a residual error as
%
\begin{equation}
    \bfe(\bfx_i, \bfx_m, \bias_i) 
    = \D_{im}(\bias^i, \bfx^i) \ominus \hat\D_{im}(\bfx^i, \bfx^m) =
    \begin{bmatrix}
    \Dp_{im} - \hat\Dp_{im} \\ 
    \Dv_{im} - \hat\Dv_{im} \\ 
    \Log(\hat\DR_{im}\inv \DR_{im}) 
    \end{bmatrix}.
    \label{eq:error_naive_preintegration}
\end{equation}
%
where $\ominus$ is the composite manifold lift operator defined in \ref{eq:composite_retract}.

$\hat\D_{im}$ only depends on state variables and, thus, is cheap to compute. It corresponds to the "expected" motion of the system, given the current state estimates. 
$\D_{im}$ is the motion computed from the integration of (very many) IMU measurements during $\Dt_{im}$. However, since we "naively" integrated
in the world frame, this term also depends on the initial state $\bfx_i$ and on IMU bias $\bias_i$. This implies that for each update of the estimate $\bfx_i$, the IMU measurements need to be re-integrated from the new $\bfx_i$ and for the new $\bfb^i$. This is highly inefficient and, therefore, not well adapted to the repeated evaluations required by nonlinear solvers.

To solve this problem, we need to re-define the deltas so that $\D_{im}$ is independent of the estimated states, that is, only dependent on the measured data. This new definition reads \cite{lupton-09, forster2015imu} (proof in the annex \secRef{sec:forster_proof}):
%
\begin{align}
    \D_{im}(\bias^i) \triangleq 
    \begin{bmatrix}
    \sum_{k=i}^{m} \Big[\Dv^{ik}\dt +  \frac{1}{2} \DR^{ik} (\accm{}{}^k - \bias_{\acc{}{}}^i)\dt^2 \Big] \\
    \prod_{k=i}^{m} \DR^{ik} \Exp((\accm{}{}^k - \bias_{\acc{}{}}^i)\dt)  \\
    \prod_{k=i}^{m} \Exp((\angvelm{}{}^k - \bias_{\angvel{}{}}^i)\dt)  
    \end{bmatrix}
    \label{eq:imu_delta}
\end{align}
%
and:
%
\begin{align}
    \hat\D_{im}(\bfx^i, \bfx^m) \triangleq 
    \begin{bmatrix}
    \Rot{}{}^{i,T}(\posi{}{}^m - \posi{}{}^i - \vel{}{}^i \Dt^{im} - \frac{1}{2} \grav {\Dt^{im}}^2) \\
    \Rot{}{}^{i,T} (\vel{}{m} - \vel{}{i} - \grav \Dt^{im})  \\
    \Rot{}{}^{i,T} \Rot{}{}^{m}  
    \end{bmatrix}.
\end{align}
%
Let us now emphasize the fact that $\D_{im}$ as defined in \eqRef{eq:imu_delta} does not depend on $\bfx_i$, contrary to \eqRef{eq:imu_delta_naive}. 

But we are not over yet. The dependency on the bias variable $\bfb^i$ in \eqRef{eq:imu_delta} still enforces the repeated reintegration of the measurements buffer to get $\D_{im}$ each time the solver produces a new estimate of $\bfb^i$. 
Fortunately, because the variations in $\bias_i$ are small, a linearized approximation can be used. The IMU measurements can be pre-integrated using the prior bias estimation at time $t_i$, that we note $\ol\bfb_i$, to give the pre-integrated delta $\ol\D_{im}\triangleq\D_{im}(\ol\bfb_i)$.
Then, each time a new $\bias_i$ value is computed, we can correct the delta linearly:
%
\begin{align}
    \D_{im}(\bias^i) = \ol\D_{im} \oplus \mjac{\D_{im}}{\bfb_i}(\bfb_i-\ol\bfb_i).
\end{align}
%
that is, without the need of re-integration. This is why this method takes the name of delta pre-integration.

With all these considerations, our residual error can be written as
%
\begin{align}
    \bfe_{im}(\bfx^i, \bfx^m, \bias^i) = (\ol\D_{im} \oplus \mjac{\D_{im}}{\bfb_i}(\bfb_i-\ol\bfb_i) ) \ominus \hat\D_{im}
    \label{eq:preint_residual}
\end{align}
%
In this expression, $\ol\D_{im}$ only depends on data and has been integrated only once. The rest of the operations are small and can be made as many times as necessary in the solver side.

These two observations were first made by Lupton \cite{lupton-09}, whose formulation relied on Euler angles, and were later formalized on \SO(3) using Lie theory
by Forster \cite{forster2017-TRO}. 

We also have to include a factor on the successive bias estimates taking into account the bias drift modelled as the integration of a random walk during $\Dt_{im}$:
%
\begin{equation}
    \bfe^{\bias}_{im}(\bias^i, \bias^m) = \bias^m - \bias^i
\end{equation}
%
with associated covariance matrix
%
\begin{equation}
    \Cov^{\bias}_{im} = 
    \begin{pmatrix}
    \bfW^c_{\angvel{}{}} \Dt_{im} & \bf0 \\
    \bf0 & \bfW^c_{\acc{}{}} \Dt_{im}
    \end{pmatrix}
\end{equation}

We will now show how this formulation can be generalized to other high rate sensory data by abstracting a recursive implementation of the pre-integration that takes advantage of the Lie-group structure of the geometry of the deltas.



\section{Generalized pre-integration on Lie groups}
\label{sec:general-preint}


Pre-integration refers to the integration of high rate proprioceptive sensory data in an efficient way in the context of factor graphs. 
As we have seen for IMU measurements, if a standard integration is conducted naively to derive a factor measurement between two \keyframes, 
then the data needs to be reintegrated at each solver iteration because the integral depends on the states we are to estimate. 
The IMU pre-integration theory solves this issue and can be generalized to many other proprioceptive sensors as shown in \cite{atchuthan-18-thesis,deray-19-selfcalib,fourmy2021contact}. 

For any motion sensor which data measurements we integrate, we need to find the appropriate delta quantities $\D_{im}$ formulation so that there exists an operator $\boxplus$ (and its inverse $\boxminus$) that verify
%
\begin{align}
    \boxplus~&: \quad\quad \cM_{\bfx} \times \cM_{\D} \rightarrow \cM_{\bfx}; 
    \quad\quad (\bfx_i,\D_{im}) \rightarrow \bfx_m=\bfx_i\boxplus \D_{im} \\
    \boxminus~&: \quad\quad \cM_{\bfx} \times \cM_{\bfx} \rightarrow \cM_{\D}; 
    \quad\quad (\bfx_i,\bfx_m) \rightarrow \D_{im}=\bfx_m\boxminus \bfx_i~.
\end{align}
%
Suitable $\D_{im}$ and $\boxplus$ have to be chosen so that the measurement integration $\D_{im}$ does not depend on the initial states $\bfx_i$.

Thanks to the geometry of motion, such deltas have a group structure: they form a Lie group (see \secRef{sec:lie_groups}).
The Delta Lie group used in the specific application has to be defined, by describing in particular its identity element $\D_{\cE}$, the composition law $\circ$, and the $\Exp$ and $\Log$ operators. Likewise, in order to deal with the uncertainty of the integrated data, Jacobians of these operators will need to be obtained. 

The motion data usually expresses the way these deltas change, and can therefore easily be put in the tangent space of the Deltas Lie group. In order to account for sensor self-calibration, this motion data is first corrected for using known calibration parameters. The resulting tangent vector is then retracted to the Lie group using the exponential map, then composed with the delta pre-integrated so far to obtain, recursively, a new value of the pre-integration.


%The motion measurement at time $k$, $\bfz_k$, is the velocity of these deltas: it belongs to the tangent space of the group of deltas. 
% The motion measurement may be biased (such as in the IMU case \eqRef{eq:imu_meas_model}), which forces us to include extra parameters as estimation variables.
In some cases, using composite Lie groups makes the math easier than using compact Lie groups. For the IMU, we explored the two variants: composite \cite{forster2015imu} and compact group (our paper \cite{fourmy2019absolute}). In \chpRef{chp:underactuade_dynamics}, we present an application of this theory to the pre-integration of force-torque sensors for the centroidal estimation of legged robots based on the composite Lie group approach. This work was part of our published paper \cite{fourmy2021contact}.

Dependence on other small-varying parameters $\bfb$ such as sensor bias or other calibration parameters is linearized as $\D_{im}(\bfb)=\D_{im}(\ol\bfb) \op \bfJ^\D_\bfb(\bfb-\ol\bfb)$.
Then we can pre-integrate $\ol\D_{im}\te\D_{im}(\ol\bfb)$ once during data gathering, and use it to define residuals that are later evaluated many times by the optimizer. 

We will now show how to recursively compute the preintegrated delta $\ol\D_{im}$, the Jacobian $\bfJ^\D_\bfb$, and the covariance of $\D_{im}$, $\Cov_{\D}^{im}$.




\subsection{Delta pre-integration}

\begin{figure}[tb]
    \centering
    \includegraphics[width=0.6\textwidth]{figures/delta_time}
    \caption{The pre-integrated delta $\D_{ij}$ contains all motion from the last KF at time $i$, up to time $j$. 
    The current delta $\D_k$ contains the motion from times $j$ to $k$, computed from the last IMU measurement at time $k$.}
    \label{fig:delta_time}
\end{figure}

We perform pre-integration incrementally as follows. 
First, $\ol\D_{ii}$ is initialized to the null motion via the identity of the group, $\D_{\cE}$. 
Its covariance $\Cov_\Delta^{ii}$ and the Jacobian $\mjac{\D_i}{\bfb_i}$ are set to zero. 
At each reception of sensor data $\tilde\bfz_k$ at $t_k$, we integrate during $\dt$ to obtain the delta corresponding to a single data sample
%
\begin{equation}
    \bm\delta_k = f(\tilde\bfz_k, \ol\bfb_i, \dt)~, 
    \label{eq:data2delta}
\end{equation}
%
using the bias $\ol\bfb_i$ available in KF $i$. 
In major cases, this function is split into the stages of data calibration $c()$, producing a vector in the Lie algebra, and retraction $\Exp()$, retracting it onto the manifold,
%
\begin{align}
    \bftau_k = c(\tilde\bfz_k, \ol\bfb_i) \dt \in T_\Epsilon \mathcal{M}\\
    \bm\delta_k = \Exp(\bftau_k) \in \mathcal{M},
\end{align}
%
where the former, $c()$, depends on the sensor model, and the second, $\Exp()$, on the deltas Lie group.
This single delta is integrated onto the delta pre-integrated so far using the delta composition law
%
\begin{equation}
    \ol\D_{ik} = \ol\D_{ij} \circ \bm\delta_k 
    \label{eq:deltaPlusDelta}
\end{equation}
%
as can be seen in figure \figRef{fig:delta_time}.
%with Jacobians $\mjac{\ol\D_{ik}}{\ol\D_{ij}}$ and $\mjac{\ol\D_{ik}}{\bm\delta_k}$. 
The pre-integrated delta covariance, as well as the Jacobian of the pre-integrated delta with respect to biases, are also pre-integrated using the chain rule and standard covariance propagation,
%
\begin{align}
    \Cov_\Delta^{ik} &= \mjac{\D_{ik}}{\D_{ij}}\Cov_\Delta^{ij}\mjac{\D_{ik}}{\D_{ij}}\tr 
    + \mjac{\D_{ik}}{\bm\delta_k}\mjac{\bm\delta_k}{\bfz_k}\Cov_\bfz\mjac{\bm\delta_k}{\bfz_k}\tr\mjac{\D_{ik}}{\bm\delta_k}\tr
    \\
    \mjac{\D_{ik}}{\bfb_i} &= \mjac{\D_{ik}}{\D_{ij}}\mjac{\D_{ij}}{\bfb_i} 
    + \mjac{\D_{ik}}{\bm\delta_k}\mjac{\bm\delta_k}{\bfb_i} ,
\end{align}
%
where $\mjac{\bm\delta_k}{\bfb_i}$, $\mjac{\bm\delta_k}{\bfz_k}$ are the Jacobians of \eqRef{eq:data2delta} and $\mjac{\D_{ik}}{\D_{ij}}$, $\mjac{\D_{ik}}{\bm\delta_k}$ the Jacobians of \eqRef{eq:deltaPlusDelta}, computed according to Lie theory \cite{sola2018micro}, to which we give a brief introduction in \secRef{sec:lie_groups}.



\subsection{Residual definition}
\label{sec:preint_residual}

The successive $\ol\D_{ik}$, $\mjac{\D_{ik}}{\bfb_i}$, and $\Cov_\Delta^{ik}$ are kept in a buffers initialized at $t_i$. When a new \keyframe\ is added to the problem at time $k=m$,
a factor is created to which we pass $\ol\D_{im}$, $\ol\bfb_i$,  $\mjac{\D_{im}}{\bfb_i}$, and $\Cov_\Delta^{im}$, with which the residual can be defined:
%
% The pre-integrated $\ol\D_{im}$ is used at the end of the pre-integration to define the residual:
%
\begin{align}
    \bfe_{im}(\bfx^i, \bfx^m, \bias^i) = (\ol\D_{im} \oplus \mjac{\D_{im}}{\bfb_i}(\bfb_i-\ol\bfb_i) ) \ominus \hat\D_{im}
    \label{eq:preint_residual}
\end{align}
%
with associated covariance $\Cov_\Delta^{im}$.
Here, $\bfb_i$ is the current  value of the sensor's calibration parameters,  $\hat\D_{im}=\bfx^m\boxminus\bfx^i$ is the expected delta between KFs, and $\{\op,\om\}$ are the 
 plus and minus operators described in section \secRef{sec:manifold_structure}. 
% That is, $\{\op,\om\}$ are $\{+,-\}$ for vectors, and for rotations we have $\bfR\op\bm\theta\te\bfR\Exp(\bm\theta)$ and $\bfR_2\om\bfR_1\te\Log(\bfR_1\tr\bfR_2)$. 
% The residual clearly depends on the KF states $\bfx^i,\bfx^m$ and the bias $\bfb_i$. It has an associated covariance  $\Cov_\Delta^{im}$.

In cases where the calibration parameters are subject to drift, as it is the case for IMU biases, the calibration drift is modelled as the integration of a random walk. This drift is included as a factor in the factor graph by defining a residual error
%
\begin{equation}
    \bfe^{\bias}_{im}(\bias^i, \bias^m) = \bias^m - \bias^i
\end{equation}
%
with associated covariance matrix
%
\begin{equation}
    \Cov_{\bias,im} = \bfW^c_{\bias} \Dt_{im}
\end{equation}


%
%
%
%
\section{IMU pre-integration on Lie groups}
In \cite{fourmy2019absolute}, we showed that an alternative formulation of IMU preintegration on Lie groups was possible.
We introduce a new matrix Lie group representation of the IMU deltas. The complete IMU pre-integration theory,
including the computation of the residual, is based on this new Lie structure. The theoretical material for the Lie
development in this section can be found in the report \cite{sola2018micro}.

\subsection{IMU pre-integraton on composite Lie group}
\label{sec:imu_preint_composite}

Let us come back to the IMU pre-integration problem, as stated by Forster \cite{lupton-09, forster2015imu}, defined in \secRef{sec:imu_preint_motivation}, and show that we can rewrite the algorithm in terms of the generalized pre-integration described in \secRef{sec:general-preint}.

The states involved in this integration are the base states $\bfx = [\posi{}{}, \bfv, \Rot{}{}]$ with deltas $\D=[\Dp,\Dv,\DR] \in \cM_{\D}$. 
The IMU produces biased and noisy measurements $\tilde\bfz = [\tilde\bfa, \tilde{\bfomega}]$ of the base proper acceleration and angular velocity, 
with bias $\bfb = [\bias_{a}, \bias_{\omega}]$ and noise $\noise = [\noise_{a}, \noise_{\omega}]$. 

\subsubsection{Definition of the delta operations}

The group composition law $\D^{ik}=\D^{ij}\circ\bm\delta^k$ in \eqRef{eq:deltaPlusDelta} is defined as
%
\begin{equation} 
    \label{equ:composition_delta}
    \D \circ \D
    =
    \begin{bmatrix}
        \Dp^{ij} + \Dv^{ij}\dt + \DR^{ij}\dpp^{k} \\
        \Dv^{ij} + \DR^{ij}\dv^{k} \\
        \DR^{ij}\dR^{k} 
    \end{bmatrix}
\end{equation}
%
with a group identity element composed of the identity element of its Lie groups:
%
\begin{equation}
    \D_{\cE} = [\bf0_3, \bf0_3, \bfI_3],
\end{equation}
%
its the resulting inverse being
%
\begin{equation}
\label{equ:inverse_delta}
    \D^{-1} =     \begin{bmatrix}
    - \DR\tr(\Dp + \Dv \dt) \\
    - \DR\tr \Dv \\
      \DR\tr
    \end{bmatrix}
\end{equation}
%


The pre-integration method in  \cite{forster2017-TRO} can be put in the general pre-integration formalism above by defining 
$\bm\delta=f(\tilde\bfz,\bfb,\dt)$ in \eqRef{eq:data2delta} as:
%
\begin{align}
    \bm\delta^k = \begin{bmatrix}
    \delta\bfp\\\delta\bfv\\\delta\bfR
    \end{bmatrix}^k =
    \begin{bmatrix}
    \tfrac12(\tilde\bfa-\bfb_a-\noise_a)\dt^2 \\
    (\tilde\bfa-\bfb_a-\noise_a)\dt \\
    \Exp((\tilde{\bfomega}-\bfb_\omega-\noise_\omega)\dt)
    \end{bmatrix}^k
    \label{eq:delta_function}
\end{align}

Then, $\bfx_k=\bfx_i\boxplus\D_{ik}$ is \cite[eq.~32]{forster2017-TRO} and $\D_{ik}=\bfx_k\boxminus\bfx_i$ is \cite[eq.~33]{forster2017-TRO}. 
Full details can be found in \cite[Section 3.4]{atchuthan-18-thesis}.

\subsubsection{Interpretation of the IMU deltas}

\begin{figure}[h]
\centering
\includegraphics[width=0.8\textwidth]{figures/fff}
\caption{The free-falling, non-rotating frame $\cG_t$ follows a parabolic trajectory governed only by gravity $\bfg$ and determined by the initial conditions $\bfp_i$, $\bfv_i$ and $\bfR_i$ at time $i$ ($\cG_i=\bfx_i$, blue). The IMU delta $\D_{ij}$ between times $i$ and $j$ is defined as the state of the IMU at time $j$ ($\bfx_j$, red) expressed in the free-falling frame at time $j$ ($\cG_j$, green).}
\label{fig:fff}
\end{figure}

The IMU deltas, as introduced in \cite{lupton-09, forster2015imu} can be interpreted \cite{atchuthan-18-thesis} as the motion increments, in terms of position, velocity and orientation, between the current IMU frame and another frame, that started at the IMU state at time $i$, $\bfx_i=(\bfp_i,\bfv_i,\bfR_i)$, and falls freely and without rotating at the acceleration of gravity (\figRef{fig:fff}).



\subsection{IMU pre-integration on compact Lie group}
\label{sec:imu_preint_compact}

This formulation was proposed in our published paper \cite{fourmy2019absolute}.
We propose a matrix form of the Lie group of IMU deltas as,
%
\begin{align}\label{equ:delta_Lie}
    \D &= 
    \begin{bmatrix}
    \DR & \Dv & \Dp \\
    \bf0 & 1 & \Dt \\
    \bf0 & 0 & 1
    \end{bmatrix} \in \cD \subset \bbR^{5\times 5}
    ~.
\end{align}

Group composition, identity and inverse stem from regular matrix product, identity, and inverse (with $\DR\inv=\DR\tr$).

\begin{align}
    \D\cdot\bm\delta 
    &= 
    \begin{bmatrix}
    \DR\dR & \Dv + \DR\dv & \Dp+\Dv\dt + \DR\dpp \\
    \bf0 & 1 & \Dt+\dt \\
    \bf0 & 0 & 1
    \end{bmatrix}
    ~.
    \label{eq:compact_composition}
    \\
    \D_\cE&=\begin{bmatrix}
    \bfI_3 & \bf0 & \bf0 \\
    \bf0 & 1 & 0 \\
    \bf0 & 0 & 1 
    \end{bmatrix} = \bfI_{5\times5}
    \label{equ:identity}
    \\
    \D\inv &= \begin{bmatrix}
    \DR\tr & -\DR\tr\Dv & -\DR\tr(\Dp-\Dv\Dt) \\
    \bf0 & 1 & -\Dt \\
    \bf0 & 0 & 1
    \end{bmatrix} 
    \label{equ:inverse}
\end{align}
%
Comparing against \eqsRef{equ:composition_delta}{equ:inverse_delta} we observe that this matrix Lie group behaves equivalently to Forster's IMU deltas above.


\subsubsection{Lie algebra \texorpdfstring{$\mathfrak{d}$}{d} and exponential map}

The compact Lie group defines a different Lie algebra parametrization and therefore a different exponential map.
The Lie algebra elements $\bftau\hhat$ and their isomorphic Cartesian $\bftau$ have the forms
%
\begin{align}
    \label{equ:lie_algebra}
    \bftau\hhat &= \begin{bmatrix}
    \hatx{\bth} & \bfrho & \bfupsilon \\
    \bf0 & 0 & \Dt \\
    \bf0 & 0 & 0
    \end{bmatrix} \in \mathfrak{d}
    ,&
    \bftau &= \begin{bmatrix}
    \bfrho \\ \bfupsilon \\ \bth \\ \Dt
    \end{bmatrix}
    \te \Dt \begin{bmatrix}
    \bfv \\ \bfa \\ \bw \\ 1
    \end{bmatrix} 
    \in \bbR^{10}
    ,
\end{align}
%
with $\bfv\te\dot\Dp$, $\bfa\te\dot\Dv$ and $\hatx{\bw}\te\DR\tr\dot\DR$.
Operators $\wedge$ and $\vee$ are defined so that $\bftau\hhat=(\bftau)\hhat$ and $\bftau=(\bftau\hhat)\vvee$.

The exponential map transfers tangent elements to the group; the logarithmic map is its inverse,
%
\begin{align}
    \D &= \Exp(\bftau) \te \exp(\bftau\hhat) = \begin{bmatrix}
    \Exp(\bth) & \bfQ\bfupsilon & \bfQ\bfrho+\bfP\bfupsilon\Dt \\
    \bf0 & 1 & \Dt \\
    \bf0 & 0 & 1
    \end{bmatrix} %\in\cD 
    \label{eq:compact_exp}
    \\
    \bftau &= \Log(\D) \te \log(\D)\vvee = \begin{bmatrix}
    \bfQ\inv(\Dp-\bfP\bfQ\inv\Dv\Dt) \\
    \bfQ\inv\Dv \\
    \Log(\DR) \\
    \Dt 
    \end{bmatrix} %\in \bbR^{10}
    \label{eq:compact_log}
\end{align}
%
where $\Log()$ is obtained by identifying terms in \eqRef{equ:delta_Lie} and \eqRef{eq:compact_exp}.
Matrices $\bfP$ and $\bfQ$ are provided in the appendix's \secRef{sec:IMULieGroup}.


\subsubsection{The IMU case of \texorpdfstring{$\bfv=0$}{bfv=0}}
\label{sec:IMU_case}

We showed that IMU deltas could be interpreted as the motion relative to the free-falling frame (\figRef{fig:fff}), which has initial velocity $\bfv_i$, and the same physical analogy applies to the compact Lie group.
Thus the tangent velocity $\bfv=\dot\Dp$  is zero at the start of the integration step. 
%
Since the exponential $\Exp(\bm\nu\Dt)$ assumes a constant tangent vector $\bm\nu=(\bfv,\bfa,\bw,1)$ during the interval $\Dt$, we have that $\bfv=0$ during the full step. 
%
This gives immediately
%
\begin{equation}
    % \D &= 
    \Exp\left(\begin{bmatrix}
    \bf0 \\ \bfa \\ \bw \\ 1
    % \hatx{\bw} & \bfa & \bf0 \\
    % \bf0 & 0 & 1 \\
    % \bf0 & 0 & 0 
    \end{bmatrix}\Dt\right) 
    = \begin{bmatrix}
    \Exp(\bw\Dt) & ~\bfQ\bfa\Dt~ & \bfP\bfa\Dt^2 \\
    \bf0 & 1 & \Dt \\
    \bf0 & 0 & 1
    \end{bmatrix}
    ~,
    \label{eq:exp_compact_imu}
\end{equation}
%
which we will use to integrate IMU data.


\subsubsection{Jacobians, uncertainty}
\label{sec:uncertainty}

For general functions $f:\cM\to\cN;y=f(x)$, we propagate uncertainty normally via the Jacobians 
$\mjac{y}{x}\te\dpar{y}{x}$, \ie, $\Cov_y=\mjac{y}{x}\,\Cov_x\,\mjac{y}{x}\tr$. 
These Jacobians map the tangent spaces of the manifolds $\cM,\cN$ at $x$ and $y$, and in case of vector spaces they resort to the classical Jacobian.
They also satisfy the chain rule, which we use extensively in our developments.
Ample reference and justification of this approach can be found in \cite{sola2018micro}.

A comment is however necessary for the present IMU case.
It relates to the uncertainty of the last component of the tangent space \eqRef{equ:lie_algebra}, which is the time $\Dt$. This component has no uncertainty by definition. 
Having it in the covariances would imply singularity and result in the risk of a number of well-known numerical issues. 
We therefore systematically marginalize this time component out of the covariances, simply by removing the last row and column. 






%
%
%
\subsubsection{Pre-integration pipeline}

Here we recall the general pre-integration pipeline in the case of compact Lie group, pointing at the minor differences with composite Lie group case.
The following pipeline of operations is performed to recursively pre-integrate IMU data into a unique measurement.

At the reception of each IMU measurement $\bfz_k=(\bfa,\bw)_k$, start by correcting it with available  bias estimates $\ol\bfb_i=(\bfa_b,\bw_b)_i$, 
to produce the tangent vector $\bftau=\bm\nu\dt$. For this, set the velocity part of $\bm\nu$ to zero as the IMU is by definition at zero speed \wrt 
the moving frame, as in \eqRef{eq:exp_compact_imu}. Obtain at the same time the respective Jacobians,
%
\begin{align}
    \bftau_k &= 
    \begin{bsmallmatrix}
    \bf0 \\ \bfa-\bfa_b \\ \bw-\bw_b \\ 1
    \end{bsmallmatrix}\dt, 
    &
    \mjac{\bftau}{\bfz} &= 
    \begin{bsmallmatrix}
    \bf0 & \bf0 \\
    \bfI & \bf0 \\
    \bf0 & \bfI \\
    0 & 0
    \end{bsmallmatrix}\dt,
    &
    \mjac{\bftau}{\bfb} &= 
    -\begin{bsmallmatrix}
    \bf0 & \bf0 \\
    \bfI & \bf0 \\
    \bf0 & \bfI \\
    0 & 0
    \end{bsmallmatrix}\dt 
    \label{eq:preint_debiasing}
\end{align}
%
Second, use the exponential map \eqRef{eq:exp_compact_imu} to obtain the current delta step $\bm\delta_{jk}$ in the group manifold, and obtain Jacobian
%
\begin{align}
    \bm\delta_{jk} &= \Exp(\bftau_{k})~,
    &
    \mjac{\bm\delta}{\bftau} &= \mjac{}{r}(\bftau_k)
    ~.
    %
    \intertext{Third, use group composition \eqRef{eq:compact_composition} to update the pre-integrated delta; obtain Jacobians}
    %
    \ol\D_{ik} &= \ol\D_{ij}\circ\bm\delta_{jk} ~,
    &
    \mjac{\D_{ik}}{\D_{ij}} &= \Ad{\bm\delta_{jk}}\inv
    ~,&
    \mjac{\D_{ik}}{\bm\delta_{jk}} = \bfI
    \label{equ:pre_composition}
    ~,
\end{align}
%
where $\Ad{\bm\delta}$ is the adjoint and $\mjac{}{r}$ is the right Jacobian ---see appendices \secRef{sec:imu_compact_adjoint}, \secRef{sec:imu_compact_right_jacobian} and technical paper \cite{sola2018micro} for reference. Fourth, propagate the delta covariance
%
\begin{align}
    \bfSigma^\Delta_{ik} &= \mjac{\D_{ik}}{\D_{ij}}\bfSigma^\Delta_{ij}\mjac{\D_{ik}}{\D_{ij}}\tr 
    + \mjac{\D_{ik}}{\bfz}\bfSigma_\bfz\mjac{\D_{ik}}{\bfz}\tr
    ~,
    %
    \intertext{with $\bfSigma_\bfz$ the covariance of the IMU measurements $\bfz$, and $\mjac{\D_{ik}}{\bfz}=\mjac{\D_{ik}}{\bm\delta}\mjac{\bm\delta}{\bftau}\mjac{\bftau}{\bfz}$ computed using the chain rule. Finally, integrate the Jacobian of the delta \wrt the biases}
    %
    \mjac{\D_{ik}}{\bfb} &= \mjac{\D_{ik}}{\D_{ij}}\mjac{\D_{ij}}{\bfb} 
    + \mjac{\D_{ik}}{\bm\delta_{jk}}\mjac{\bm\delta}{\bftau}\mjac{\bftau}{\bfb}
    ~.
\end{align}
%

Pre-integration starts after each keyframe with $\ol\D_{ii} = \bfI$, $\bfSigma^\Delta_{ii} = \bf0$ and $\mjac{\D_{ii}}{\bfb} = \bf0$, using $\ol\bfb_i = \bfb_i$ the current best estimate of the bias at time $i$.
Pre-integration is complete (\figRef{fig:delta_time}) when $k=m$, which yields $\ol\D_{im}$, $\bfSigma^\Delta_{im}$ and $\mjac{\D_{im}}{\bfb}$.


\subsubsection{Factor residual}
Computation of the residual in the case of the Lie Delta group formulation follows the steps defined in \secRef{sec:general-preint}.
% However, since the Delta is directly defined as a Lie group, the formulation is actually simplified because  ... 
Use the pre-integrated Jacobian $\mjac{\D_{im}}{\bfb}$ to correct the pre-integrated delta $\ol\D_{im}$ to account for the new bias estimate $\bfb_i\neq\ol\bfb_i$,

\begin{align}
    \D_{im}(\bfb_i) &= \ol\D_{im}\cdot\Exp(\mjac{\D_{im}}{\bfb}(\bfb_i-\ol\bfb_i)) 
    ~.
\end{align}

Use \eqRef{equ:delta} as $\boxminus$ to compute the expected delta from  $\bfx_i$ to $\bfx_m$,
%
\begin{align}
    \widehat\D_{im}(\bfx_i,\bfx_m) &= \bfx_m \boxminus \bfx_i 
    ~.
\end{align}

Compute the residual in the  tangent of $\cD$ at $\D_{im}$,
%
\begin{align}
    \bfe^\D_{im}(\bfx_i,\bfx_m,\bfb_i) 
    &=\widehat\D_{im}(\bfx_i,\bfx_m) \ominus \D_{im}(\bfb_i) \\
    &=\Log(\D_{im}(\bfb_i)\inv \cdot \widehat\D_{im}(\bfx_i,\bfx_m)) \in \bbR^9
~,
\end{align}
%
and drop the $\Dt$ part from the residual after the $\Log()$ ---see comment in \secRef{sec:uncertainty}.
In this last equation, the minus operator $\ominus$ is simply the traditional Lie group "minus" operator defined in \cite{sola2018micro} and specialized for this particular group.


\subsection{About the choice of the proper Lie group}

\subsubsection{Our IMU Lie group versus Forster's method}

Mathematically, and disregarding methodology, the main difference between our method (\secRef{sec:imu_preint_composite}) and Forster's \cite{forster2017-TRO} (\secRef{sec:imu_preint_compact}) is to be found in the exponential map. 
To see it, let us consider small rotation increments $\bth=\bw\dt$ captured at each single IMU sample. 
In such cases, the matrices $\bfP,\bfQ$ appearing in the exponential map \eqRef{eq:compact_exp} and detailed in \eqRef{equ:RQP} can be approximated by $\bfP\approx\tfrac12\bfI$ and $\bfQ\approx\bfI$.
The exponential becomes,
%
\begin{align}
    \Exp\left(\begin{bmatrix}
    \bf0 \\ \bfa \\ \bw \\ 1
    \end{bmatrix}\dt\right) \approx \begin{bmatrix}
    \Exp(\bw\dt) & \bfa\dt & \tfrac12\bfa\dt^2 \\
    \bf0 & 1 & \dt \\
    \bf0 & 0 & 1
    \end{bmatrix}
~,
\end{align}
%
where we find the terms $\bfa\dt$ and $\tfrac12\bfa\dt^2$, which should sound familiar from Forster's method. 
In effect, with this approximation, if we now compact all the steps \eqsRef{eq:preint_debiasing}{equ:pre_composition} of our integration into a cumulative expression,
%
\begin{align}
    \D_{ik} = \prod_{j=i+1}^k \Exp\left(\begin{bmatrix}
    \bf0 \\ (\bfa_j-\bfa_{bi}) \\ (\bw_j-\bw_{bi}) \\ 1
    \end{bmatrix}\dt\right)
~,
\end{align}
%
it is possible (although tedious) to show that both Forster's and our method are exactly equivalent when $\bw\dt\to0$.
% , which is usually a valid hypothesis.
% These differences should not constitute an argument against Forster, since in practice we typically have extremely small steps $\bw\dt$ and the approximation holds very well.

\subsubsection{Further discussion regarding Lie group choice}

Regarding "compact group" designs, there is still another proposal, the $\SE_2(3)$ group proposed by \cite{barrau2020mathematical, brossard2021associating}, 
which can be used for IMU pre-integration. This proposal differs from ours in the following aspects:

\begin{itemize}
    \item The $\SE_2(3)$ group does not contain the time and therefore the composition law does not account for the whole integration. Some extra algebra needs to be added.
    \item The $\SE_2(3)$ group is easier to manipulate since the closed forms for the exponential map, the adjoint and the right-jacobian are easier to obtain.
    \item Our group better separates between the delta states and the velocity of these states which depend only on the IMU data.
\end{itemize}

Therefore, it is apparent that depending on the structure of the defined Lie group, we can have different designs:
\begin{itemize}
    \item Forster \cite{forster2015imu}: Composite Lie group
    \item Barrau \cite{barrau2020mathematical}: $\SE_2(3)$ compact Lie group without time
    \item Fourmy \cite{fourmy2019absolute}: Compact IMU group with time
\end{itemize}

In the context of filtering, using a compact group formulation of the robot state has been proven to improve greatly the basin of convergence of the so-called
Invariant Kalman Filter \cite{barrau2018invariant, hartley2020contact} by avoiding the need to compute jacobians around the current estimates (which is done
with the standard EKF), which may lead to an inconsistent estimate if the filter is initialized far from the optimal. 
However, this issue is not so present with factor graph optimization since we repeatedly linearize around the new estimate.
Nevertheless, it seems that compact Lie groups may provide slightly better performances than composite Lie groups \cite{brossard2021associating}, thanks 
to a more precise linearization \wrt\ the IMU biases and a covariance propagation that better represents the geometry of the problem.

However one may ask whether these more elegant mathematical formulations and slight improvements are worth the effort. Compact designs may be seen as going against
our modularity philosophy since for each new motion sensor, we need to find a new appropriate Lie group instead of being able to reuse the machinery developed
for composite groups, which is vastly simpler.

\section{Related works}


Pre-integration principles were first proposed by Lupton \cite{lupton-09} to be applied to a smoothing based visual inertial estimator. His work was motivated partly 
by the fact that previous systems required a precise initialization of position, orientation and velocity (using a specialized routine) to begin to integrate IMU measurements. 
With this new formulation, Lupton noted that pre-integration of IMU measurements permitted to use measurements immediately and delay initial orientation about the gravity
vector in particular. 
This seminal work was quickly adopted by other authors using smoothing filters \cite{carlone2014eliminating}. As pioneering as this work was, it was however 
limited by the use of Euler angles whose problematic geometric properties are notorious. Indelman \cite{Indelman-2013-7768} first proposed to use the exponential of the 
matrix rotation group instead of Euler integration and to relax the assumptions of a non-rotating and flat earth of Lupton \cite{lupton-09}. Forster \cite{forster2015imu, forster2017-TRO}
proposed the same formulation using instead quaternions. Various experiments brought to light three main problem with the Euler angle formulation, that are completely absent 
from the quaternion "on-manifold" formulation. Firstly, first order integration of angular velocities using Euler angles is approximate, which leads to accumulated errors 
for high angular velocities or sampling rate.  Secondly, the log-likelihood of the angular displacement is not invariant under the action of rigid body transformations, 
\eg the choice of the world frame influence the results of the estimation. Finally, the well known gimbal lock singularity of Euler angles has a consequence 
on the covariance IMU noise covariance propagation, which is severely degraded when the robot trajectory comes close to the singularity. 
\cite{shen2015tightly}, later improved in \cite{qin2018vins} proposed to use a more precise numerical integration procedure than the default forward Euler used by Forster. 
Eckenhoff \cite{eckenhoff2019closed} derived closed form solutions of the preintegration equations using various piecewise constant models.

Barrau \cite{barrau2020mathematical} described a coupled matrix Lie group for the propagation of preintegration errors taking into account the earth rotation for aerospace
inertial navigation system in view. This work was later extended \cite{brossard2021associating} and showed that the linearized bias update is slightly more precise than 
the work of Forster \cite{forster2017-TRO}. Le Gentil \cite{le2020gaussian} used a different trajectory parametrization framework by formulating the preintegration algorithm 
in the context of Gaussian Process smoothing. Self calibration of IMU/Camera time offsets was also developped \cite{yang2020analytic}. 
\cite{luo2021unified} derived a comprehensive collection of motion models depending on the various possible choices of reference frames and motion conditions. 

As we saw the preintegration theory began in the context of visual-inertial odometry. It was however adapted to other high rate sensors such wheel odometry \cite{quan2019tightly}, 
possibly including self-calibration \cite{deray-19-selfcalib}. In his thesis, Atchuthan (\cite{atchuthan-18-thesis}, Section 4.3) derived the general form of the preintegration 
equations as a sensor agnostic form that is integrated in state estimation WOLF \cite{sola2021wolf}. As previously mentioned, other team applied preintegration theory in the 
context of factor graph legged robot state estimation to derive new leg odometry factors \cite{hartley2018legged, wisth2019robust, wisth2020preintegrated}.
It was also applied to integrate drone dynamics to estimate external forces disturbances \cite{nisar2019vimo}.

In \chpRef{chp:underactuade_dynamics}, we propose to pre-integrate force-torque measurements that are present in some legged-robots. We show that permits to estimate the centroidal quantities of the robot as well making the kinematic bias on center of mass measurements observable. A practical implementation is demonstrated in \ref{chp:centroidal_estimation}, which is based on our paper \cite{fourmy2021contact}

