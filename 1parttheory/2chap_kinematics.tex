\chapter{Kinematics}
\minitoc

The robot kinematic model along with encoder measurements make possible the computation of poses and spatial velocities \cite{featherstone2014rigid} (aka. twists) 
of reference frames attached to the robot segments relative to the base or world frame. We will shortly describe the computation of these quantities and their 
application to leg odometry.


\section{Forward kinematics}
The number of freedom (DoF) $n$ of a poly-articulated system is the minimal number of variables that completely describe his state given the base frame
pose and spatial velocity. For robots with rigid segments, the DoF of the robot is the sum of the DoFs of its joints (1 for linear and rotational joints, 3 for
ball joints, etc.). The state of the joints (1 angle for rotational joints) can be stacked together in the vector of joint configurations 
$\qa \in \Reals$. For most simple joins, the joint configuration velocities are simply the time derivative $\dqa \in \Reals^n$\footnote{An exception is the ball joint where the 
state is defined as an element of $\SO(3)$, thus its velocity vector is defined as an angular velocity in $\Reals$.}. $\qa$ and $\dqa$ are typically obtained from
the joint encoder measurements.

The whole body state of the robot is defined by the robot configuration vector $\bfq$ and configuration velocity vector $\bfv_q$. Those are concatenations
of the base state and joint configuration vectors:
%
\begin{equation}
    \bfq=
    \begin{pmatrix}
        \posi{w}{b} \\
        \prescript{w}{}{\bfq}_b \\
        \qa
    \end{pmatrix} \in \Reals^{7+n},
    \quad \quad
    \bfv_q=
    \begin{pmatrix}
        \prescript{b}{}{\mathbf{\nu}}_b  \\
        \dqa
    \end{pmatrix}
    \label{eq:configuration} \in \Reals^{6+n}
\end{equation}
%
where $\posi{w}{b} \in \Reals^3$ is the position of the base relative to the world expressed in world frame, $\prescript{w}{}{\bfq}_b \in \Reals^4$ is the orientation 
of the base frame with respect to the world as a unit quaternion, and $\prescript{b}{}{\mathbf{\nu}}_b$ is the twist of the base frame expressed in the base frame.
The computation of any segment pose of index $k$ relative to the world is obtained using the forward kinematics (FK) algorithm:
%
\begin{equation}
    \T{w}{k} = \text{FK}(\bfq, \text{segment k id})
\end{equation}

It is also possible to obtain the pose relative to the base $\T{b}{k}$ by simply setting the base pose vector to be the identity pose $[0,0,0,~0,0,0,1]$.
FK is a nonlinear function of the robot state. To obtain the relative spatial velocity, we have to compute the jacobian $\bfJ_k$ of FK with respect to the configuration.
The differential forward kinematics (DFK) is then simply:

\begin{equation}
    \prescript{w}{}{\mathbf{\nu}}_k = \bfJ_k \bfv_q    
\end{equation}
Similarly, the spatial velocity relative to the base $\prescript{b}{}{\mathbf{\nu}}_k$ can be obtained by setting the base spatial twist part of $\bfv_q$ to zero.
These algorithms are very fast to compute ($\approx 1\mu s)$ using modern dedicated libraries \cite{carpentier2019pinocchio}.

We will now see how these base algorithms can be used to derive leg odometry for a quadruped robot.

\section{Leg odometry}
We will restrict ourselves to the case of point feet as the main platform on which we experimented is a quadruped robot.
Quadrupeds are usually equipped with nonarticulated round feet whose contact with the ground, is in first approximation, punctual.
Humanoid robots are equipped with articulated flat feet that provide richer information. In \ref{sec:kinematic_info} we gave an exhaustive review of the types of leg 
odometry measurements that can be obtained on legged machines.

The basic idea is that we assume to have access to accurate contact detection. If the contact is held between time $t_i$ and $t_j$ and that
the contact point $L$ (to which a frame is attached) is fixed. This can be written $\posi{W}{L}^i = \posi{W}{L}^j$. 
In practice, the round feet of our robot can slightly roll but we will neglect this phenomenon for now. 
If we unroll the transformation chain to make the base poses appear in the equation, we obtain the identity:

\begin{equation}
    \posi{}{}^i + \Rot{}{}^i \posim{B}{L}^i = \posi{}{}^j + \Rot{}{}^j \posim{B}{L}^j
\end{equation}

To turn this equation into a measurement model, we need to model the influence of measurement inaccuracies.
This equation depends on relative positions $\posim{B}{L}^i$ and $\posim{B}{L}^j$ that are computed using FK. The quality of the computation depends on several things:
the robot kinematic model (calibration), hard to model deviation from the rigid body model (\eg segment/joint flexibilities, backlash), encoder noise.
It is tempting to propagate encoder noise $\noise_{\qa}$ trough FK using the kinematic jacobian \cite{bloesch2013state, hartley2018legged}:

\begin{equation}
    \posim{B}{L} = FK(\qa + \noise_{\qa}) \approx FK(\qa) + \bfJ_L\noise_{\qa}, \quad \quad \Cov_{p} = \bfJ_L \Cov_{\qa} \bfJ_L
\end{equation}

However, the resolution of Solo encoder is actually very high: 0.002 degrees which would account for a mere 10 \textit{micrometers} difference.
Therefore, most of the measurement errors come from the other mentioned effects. Unfortunately, those are much harder to model, and all act at the same time
in an unpredictable fashion. Thus, we opt for a simple additive noise on the foot position in the world:

\begin{equation}
    \posi{W}{L}^i = \posi{W}{L}^j + \noise_{LO}
\end{equation}

where $\noise_{LO} \sim \Gaussian{0}{\Cov_{LO}}$ denotes a Gaussian noise accounting for a potential roll of the foot and kinematic inaccuracies.
This noise can be modeled as the integration of white noise on the foot velocity $\noise_v$, which integrated gives a random walk.  
Its variance after $\Dt$ is $\Cov_{LO} = \sqrt{\Dt} \Cov_v$.
The residual $\bfe^{LO}$ is then expressed for each foot $l$ in contact between $t_i$ and $t_j$:

\begin{equation}
    \bfe^{LO}_l(\posi{}{}^i, \Rot{}{}^i, \posi{}{}^j, \Rot{}{}^j) = \posi{}{}^i + \Rot{}{}^i \posim{B}{L}^i - (\posi{}{}^j + \Rot{}{}^j \posim{B}{L}^j)
\end{equation}

Where $\posim{B}{L}^i$ and $\posi{B}{L}^j$ are the contact positions in base frame at times $t_i$ and $t_j$ acquired from $\qa$ via forward kinematics. 
In the literature review and \figRef{fig:kin_models}, this method is referred to as \textit{single foot matching}. To obtain
a relative transformation between $t_i$ and $t_j$, at least three feet are necessary. When it is the case, it is possible to directly compute the
relative transformation by solving an orthogonal Procrustes problem \cite{roston1991dead} which would be integrated as a separate residual.
We did not found useful to implement such a factor for several reasons. First of all, this computation is implicitly handled by the solver.
Secondly, for trotting gates common for quadrupeds, the robot is most of the time standing on two legs. Thirdly, when fusing with other sensors (such as the IMU,
see [REF application IK for an example]), 1 or 2 legs in contact already provide sufficient information to make the desired variables observable.

% Assuming good contact detection, leg odometry fused with the IMU is enough to properly observe 6 DoF motion with as few as 2 contact feet at a time. 
% Note that contrary to previous works \cite{hartley2018legged, wisth2020preintegrated}, this leg odometry factor does not require gyroscope integration.


\section{Terrain height}
If we have prior knowledge about the foot terrain height $h \in \Reals$, it is easy to define, for each foot $l$ in contact, the residual:

\begin{equation}
    \bfe^h_l(\posi{}{}, \Rot{}{}) = (\posi{}{} + \Rot{}{} \posi{b}{l})_z - h
\end{equation}

This residual can be assigned with a variance $\sigma_h^2$ which is hand-tuned.

% \section{Kinematic calibration}
% A crucial part of the usability of this measurement model is to be able to compute an unbiased $\posi{B}{L}$ estimation for each foot.
