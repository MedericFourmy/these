\chapter{Kinematics}
\minitoc

The most legged-robot specific sensor measurements that we developed are those related to the robot kinematics. 
The principle is to use the robot kinodynamic model to compute relative placement and velocity of different parts of the robot 
with respect to the base frame. This computation is done using the vector of configuration of the robot $\qa$ and its derivative $\dqa$ which
are typically obtained from joint encoders.
For centroidal quantities, we can compute the CoM position, velocity and the angular momentum.
For leg odometry, the relative placement of the feet with respect to the base is considered. A special care has to be taken in order to
obtain non biased measurements. In particular, calibration of the robot kinematic model is of paramount importance. We left dynamic calibration 
out of the scope of this work.


\section{Centroidal kinematics}
Use of the kinematic model of the robot to obtain measurements about centroidal quantities of the robot.

We need to relate base states to centroidal quantities to ground their estimate. 
For that, we can rely on the inertial-kinematic model to compute the CoM position wrt. base frame $\posi{B}{C}(\qa) \in \Reals^3$, the CoM velocity wrt. 
base frame $\vel{B}{C}(\qa, \dqa) \in \Reals^3$, the inertial matrix $\Inertia(\qa) \in \Reals^{3\times 3}$, and  the kinematic momentum due to gesticulation 
of the robot limbs $\AM_a(\qa, \dqa) \in \Reals^{3}$. 
As stated before, the computed CoM position is considered to be affected by Gaussian noise and a slowly varying bias $\posim{B}{C} = \posi{B}{C} + \bfb_{c} + \noise_c$. 
The angular velocity from the IMU is used and its bias has to be incorporated in the factor $\angvelm{}{} = \angvel{B}{B} + \bfb_{\omega} + \noise_{\omega}$.  
In the end the equations used to derive the factor are:
%
\begin{equation}
\small
\begin{split}
\COM &= \Rot{}{} (\posim{B}{C} -  \bias_{c} - \noise_{c}) + \posi{}{}
\\
\COMd &= 
\vel{}{} + \Rot{}{}((\angvelm{}{} - \bias_{\omega} - \noise_{\omega}) \times (\posim{B}{C} -  \bias_c - \noise_c) 
\\&~~~~~~+ (\velm{B}{C} - \noise_{v}))
\\
\AM &= \Rot{}{}(\Inertia (\angvelm{}{} - \bias_{\omega} - \noise_{\omega}) + \AM_a)
\end{split}
\end{equation}

Then, the residual $\bfr^{CK} \in \Reals^9$ is simply expressed as:
%
\begin{equation}
\small
\bfr^{CK}=
\begin{bmatrix}
\posim{B}{C} - (\Rot{}{}^T(\COM - \posi{}{}) + \bias_{c})
\\
(\angvelm{}{} - \bias_{\omega}) \times (\posim{B}{C} -  \bias_{p}) + \velm{B}{C} - \Rot{}{}^T(\COMd - \vel{}{})
\\
\Inertia (\angvelm{}{} - \bias_{\omega}) + \AM_a - \Rot{}{}^T\AM
\end{bmatrix}
\label{eq:CKFactor}
\end{equation}



\section{Leg odometry}
Assuming that we have access to an accurate contact detection, for each foot in contact between \keyframes $i$ and $j$, we can write that 
$\posi{W}{L}^i = \posi{W}{L}^j + \noise_{LO}$ where $\noise_{LO}$ denotes a Gaussian noise accounting for potential slip of the foot and kinematic inaccuracies 
$\Gaussian{0}{\sigma_{LO}}$.
This noise can be modeled as the integration of a white noise on the foot velocity $\noise_v$, which integrated gives a random walk.  
Its variance after $\Dt$ is $\sigma_{LO} = \sqrt{\Dt} \sigma_v$
This pseudo measurement can be used to derive a 3D factor for each foot $l$ in contact:

\begin{equation}
    \bfr^{LO}_l(\posi{}{}^i, \Rot{}{}^i, \posi{}{}^j, \Rot{}{}^j) = \posi{}{}^i + \Rot{}{}^i \posim{B}{L}^i - (\posi{}{}^j + \Rot{}{}^j \posim{B}{L}^j)
\end{equation}

Where $\posim{B}{L}^i$ and $\posi{B}{L}^j$ are the contact positions in base frame at times $i$ and $j$ acquired from $\bfq_a$ via forward kinematics. 

Assuming good contact detection, leg odometry fused with the IMU is enough to properly observe 6 DoF motion with as few as 2 contact feet at a time. 
Note that contrary to previous works \cite{hartley2018legged, wisth2020preintegrated}, this leg odometry factor does not require gyroscope integration.


\section{Terrain height}

If we have prior knowledge about the foot terrain height $h \in \Reals$, it is easy to define, for each foot $l$ in contact, the residual:

\begin{equation}
    \bfr^h_l = (\posi{}{} \posi{b}{l})_z - h
\end{equation}

This residual can be assigned with a variance $\sigma_h$ which is hand tuned.

\section{Kinematic calibration}
A crucial part of the usability of this measurement model is to be able to compute an unbiased $\posi{B}{L}$ estimation for each foot.
