\chapter{Kinematics}
\minitoc
\bigskip

The robot kinematic model along with encoder measurements make possible the computation of poses and spatial velocities \cite{featherstone2014rigid} (aka. twists) 
of reference frames attached to the robot segments relative to the base or world frame. This kinesthesis capability is of paramount importance to control
multi-body systems for locomotion or other interactions with its environment. If we have information about stable contacts that the robot keeps with
its environment, it is possible to infer a relative motion measurement, which is commonly called leg-odometry.

We will shortly describe the formulation of the forward kinematics algorithm, that enables to compute relative poses and spatial velocities between parts of the robot
and then show how this information can be used to define a leg-odometry factor. The last section shows how forward kinematics can also be used to leverage prior 
information about the height of the environment. 



\section{Forward kinematics}
\label{sec:forward_kinematics}
We first describe the forward kinematics and differential kinematics algorithms by introducing notations commonly found in the legged robots literature.

The degrees of freedom (DoF) $n$ of a poly-articulated system is the minimal number of variables that completely describe his state given the base frame. 
For robots with rigid segments, the DoF of the robot is the sum of the DoFs of its joints (1 for linear and rotational joints, 3 for
ball joints, etc.).
The state of the joints (1 angle for rotational joints) can be stacked together in the vector of joint configurations 
$\qa \in \Reals^n$. For most simple joints, the joint configuration velocities are simply the time derivative $\dqa \in \Reals^n$\footnote{An exception is the ball joint where the 
state is defined as an element of $\SO(3)$, thus its velocity vector is defined as an angular velocity in $\Reals^3$.}. $\qa$ and $\dqa$ are typically obtained from
the joint encoder measurements.
Legged robots are not fixed to the ground so we add a so called "free-flyer" which is the pose of the base with respect to the world and is modeled as an unactuated joint. 

The whole body state of the robot is defined by the robot configuration vector $\bfq$ and configuration velocity vector $\bfv_q$. Those are concatenations
of the base state and joint configuration vectors:
%
\begin{equation}
    \bfq=
    \begin{pmatrix}
        \posi{w}{b} \\
        \prescript{w}{}{\bfq}_b \\
        \qa
    \end{pmatrix} \in \Reals^{7+n},
    \quad \quad
    \bfv_q=
    \begin{pmatrix}
        \prescript{b}{}{\mathbf{\nu}}_b  \\
        \dqa
    \end{pmatrix}
    \label{eq:configuration} \in \Reals^{6+n}
\end{equation}
%
where $\posi{w}{b} \in \Reals^3$ is the position of the base relative to the world expressed in world frame, $\prescript{w}{}{\bfq}_b \in \Reals^4$ is the orientation 
of the base frame with respect to the world as a unit quaternion, and $\prescript{b}{}{\mathbf{\nu}}_b$ is the twist of the base frame expressed in the base frame.
The computation of any segment pose of index $k$ relative to the world is obtained using the forward kinematics (FK) algorithm:
%
\begin{equation}
    \T{w}{k} = \text{FK}_k(\bfq)
\end{equation}

It is also possible to obtain the pose relative to the base $\T{b}{k}$ by simply setting the base pose vector to be the identity pose $[0,0,0,~0,0,0,1]$.
FK is a nonlinear function of the robot configuration. To obtain the relative spatial velocity, we have to compute the jacobian $\bfJ_k$ of FK with respect to the configuration.
The differential forward kinematics (DFK) is then simply:

\begin{equation}
    \prescript{w}{}{\mathbf{\nu}}_k = \bfJ_k \bfv_q    
\end{equation}

Similarly, the spatial velocity relative to the base $\prescript{b}{}{\mathbf{\nu}}_k$ can be obtained by setting the base spatial velocity part of $\bfv_q$ to zero.
These algorithms are very fast to compute ($\approx 1\mu s)$ using modern dedicated libraries \cite{carpentier2019pinocchio}.

We will now see how FK can be used to derive leg odometry for a quadruped robot.

\section{Leg odometry}
We will restrict ourselves to the case of point feet as the main platform on which we experimented is a quadruped robot.
Quadrupeds are usually equipped with non-articulated round feet whose contact with the ground is, in first approximation, punctual.
Humanoid robots are equipped with articulated flat feet that provide richer information. In \secRef{sec:kinematic_info} we gave an exhaustive review of the types of leg 
odometry measurements that can be obtained on legged machines.

The basic idea is that we assume to have access to accurate contact detection. If the contact is held between time $t_i$ and $t_j$ and that
the contact point $L$ (to which a frame is attached) is fixed. This can be written $\posi{W}{L}^i = \posi{W}{L}^j$. 
In practice, the round feet of our robot can slightly roll but we will neglect this phenomenon for now. 
If we unroll the transformation chain to make the base poses appear in the equation, we obtain the identity:

\begin{equation}
    \posi{}{}^i + \Rot{}{}^i \posim{B}{L}^i = \posi{}{}^j + \Rot{}{}^j \posim{B}{L}^j
\end{equation}

To turn this equation into a measurement model, we need to model the influence of measurement inaccuracies.
This equation depends on relative positions $\posim{B}{L}^i$ and $\posim{B}{L}^j$ that are computed using FK. The quality of the computation depends on several things:
the robot kinematic model (calibration), deviations from the rigid body model (\eg segment/joint flexibilities, backlash), encoder noise.
It is tempting to propagate encoder noise $\noise_{\qa}$ trough FK using the kinematic jacobian \cite{bloesch2013state, hartley2018legged}:

\begin{equation}
    \posim{B}{L} = FK(\qa + \noise_{\qa}) \approx FK(\qa) + \bfJ_L\noise_{\qa}, \quad \quad \Cov_{p} = \bfJ_L \Cov_{\qa} \bfJ_L
\end{equation}

However, the resolution of Solo encoder is actually very high: 0.002 degrees, which would account for a mere 10 \textit{micrometers} difference.
Therefore, most of the measurement errors come from the other mentioned effects. Unfortunately, those are much harder to model, and all act at the same time
in an unpredictable fashion. Thus, we opt for a simple additive noise on the foot position in the world:

\begin{equation}
    \posi{W}{L}^i = \posi{W}{L}^j + \noise_{LO}
\end{equation}
%
where $\noise_{LO} \sim \Gaussian{0}{\Cov_{LO}}$ denotes a Gaussian noise accounting for potential roll/slip of the foot and kinematic inaccuracies.
This noise can be modeled as white noise on the foot velocity $\noise_v$, which integrated gives a random walk.  
Its variance after $\Dt$ is $\Cov_{LO} = \Dt \Cov_v$.
The residual $\bfe^{LO}$ is then expressed for each foot $l$ in contact between $t_i$ and $t_j$:

\begin{equation}
    \bfe^{LO}_l(\posi{}{}^i, \Rot{}{}^i, \posi{}{}^j, \Rot{}{}^j) = \posi{}{}^i + \Rot{}{}^i \posim{B}{L}^i - (\posi{}{}^j + \Rot{}{}^j \posim{B}{L}^j)
\end{equation}
%
where $\posim{B}{L}^i$ and $\posi{B}{L}^j$ are the contact positions in base frame at times $t_i$ and $t_j$ acquired from $\qa$ via forward kinematics. 

In the literature review and \figRef{fig:kin_models}, this method is referred to as \textit{single foot matching}. To obtain
a relative transformation between $t_i$ and $t_j$, at least three feet are necessary. When it is the case, it is possible to directly compute the
relative transformation, which would be integrated as a separate residual, by solving an orthogonal Procrustes problem \cite{roston1991dead}.
We did not found useful to implement such a factor for several reasons. First of all, this computation is implicitly handled by the solver.
Secondly, for trotting gates common for quadrupeds, the robot is most of the time standing on two legs. Thirdly, when fusing with other sensors (such as the IMU,
see \chpRef{chp:centroidal_estimation}), 1 or 2 legs in contact already provide sufficient information to make the desired variables observable.

% Assuming good contact detection, leg odometry fused with the IMU is enough to properly observe 6 DoF motion with as few as 2 contact feet at a time. 
% Note that contrary to previous works \cite{hartley2018legged, wisth2020preintegrated}, this leg odometry factor does not require gyroscope integration.


\section{Terrain height}
If we have prior knowledge about the foot terrain height $h \in \Reals$, it is easy to define, for each foot $l$ in contact, the residual:

\begin{equation}
    \bfe^h_l(\posi{}{}, \Rot{}{}) = (\posi{}{} + \Rot{}{} \posi{b}{l})_z - h \quad, \quad e^{h}_l \in \Reals
\end{equation}

This residual can be assigned with a variance $\sigma_h^2$ which is hand-tuned. 
It can permit to remove vertical drift of the odometry estimation.
