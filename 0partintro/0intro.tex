\chapter{Introduction}
% De tout temps les hommes ont voulus estimer des trucs...

% ...et voila le pbe qu'on va résoudre.

%%%%%%%%%%%%%%%%%
% MEH
%%%%%%%%%%%%%%%%%
% For robots, perception of oneself and of its environment is a major challenge on the road toward many real world applications. 
% Tasks that are instinctive to us, like for instance manipulating an object, actually involve a complex 
% interactions between our many senses, our nervous and our muscular system.
% Though our sensory motor skills have been developed by millions of year of evolution, and are refined throughout our childhood,
% the task of representing them in abstracted algorithm to be implemented on a cybernetic system is a serious challenge. 
% Let's investigate the example a human lifting package. Our vision might inform us about the general form of the object, 
% its location in space with respect to us, some of its physical properties through our prior knowledge of the world. Our proprioception 
% instinctively guide our arms toward the right path. Our sense of touch might infer the surface texture of the object, 
% its softness, making us adapt our grip. During this whole process, our vestibular system
% provides us with a sense of balance to counter gravity, while our hears make us aware of events external to our current enterprise.

% All these complex phenomena happen mostly at the subconscious level while our conscious mind focuses on high level decisions.
% Imitating these skills in robot system requires then to build models of available sensor modalities and to integrate them through sensor
% fusion. This can be achieved at several levels depending on the task to solve. In the legged robot community, one of the core task is locomotion.
% For this application, robust algorithms exist in the literature using a limited set of sensors, most often inertial and contact detection.
% On the other end of the blind robot approach, a broad field of research has been concentrated on building representations of the environment 
% using exteroceptive sensors such as cameras and LIDARs. This in turn enables planning algorithms to navigate the robot in its environment. 
% Many approaches decouple the two tasks, using layered perception systems. However, theoretically, a system able to tightly fuse all the 
% available modalities would benefit a better consideration of the correlations between the different quantities to estimate. 
% Even though recent approaches have taken step in this direction, such a system is still not widely used in legged robotics. 
% This thesis is a contribution to this goal. 

%%%%%%%%%%%%%%%%%
%%%%%%%%%%%%%%%%%

Rotella 
- Legged robots benefits and challenges
- Need to estimate quantities that are not directly measurable -> data fusion
- questions about which sensors should be used for which platforms 
    -> poundering needs vs constraints (costs, weight, processing cost for embedded platforms)

Dinesh
- Need for redundant sensing -> example of living beings (inner ear, vision, kinestesis

Sola:
- State a clear objective and describe the different terms
- "hardware vs intelligence driven approaches" 
    -> more expensive hardware enable easier preprocessing (expensive IMUs = good attitude)
- very short reading guide: sketch the organisation of the thesis

Vallve:
- adapting the robot to the env rather than the other way around (motivation for autonomous systems AND legged robots)
- applications without offline maps available (because difficult to access, too expensive...) 
    -> example of the search and rescue DARPA subterran challenge
- Idea: compare DARPA 14 to DARPA 20 -> focus on real time exploration and SLAM, large scale problems

Bloesch:
- motivation for legged robots: extend range of robotic applications
- recent sucesses -> cite a bunch of recent legged robots (a lot now...)
- Central role of state est -> prerequisite for other tasks like stabilization (high bandwidth est of attitude, vel, com)
, predictive control, planning contacts etc.
- High reliability spec -> need for redundancy (Bloesch says "simple estimators" with "low complexity", quite the opposit actually)
- Short talk about inertial sensors: can provide attitude alone but fast drifting deadreckoning for position and velocity that has to be mitigated by sensor fusion.
- cameras: nice because low weight and complexity wrt LIDARS (even though now low cost depth sensor)
- vision based state est: impressive and various results. Benefits from increased robustness using IMU (only for short periods of time)
    -> shortly describe why both are very complimentary. Kinematics adds more information and increases robustness of approaches.
Remark: under represented works about semantic informations about the scene (nod to CosySLAM)
