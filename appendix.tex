\appendix

\chapter{Pre-integration details}
\section{Justification of \cite{forster2017-TRO} delta formulas}
\label{sec:forster_proof}
We detail here how the delta formulas of equation \eqRef{eq:IMUPreintij} can be derived from \eqRef{eq:IMUIntim}.

The rotation one is easy to get, multiplying by $\Rot{}{}^{i,T}$:
\begin{equation}
    \DR_{im} \triangleq \Rot{}{}^{i,T} \Rot{}{}^{m} = \prod_{k=i}^{m} \Exp((\angvelm{}{}^k - \bias_{\angvel{}{}}^k - \noise_{\angvel{}{}}^k)\dt)
    \label{eq:IMUDeltaR}
\end{equation}

The velocity is as well quite easy, by defining $\Dt_{im} \triangleq \sum_{k=i}^{m} \dt = (m-i)\dt$:
\begin{equation}
    \Dv_{im} \triangleq \Rot{}{}^{i,T} (\vel{}{m} - \vel{}{i} - g \Dt_{im}) 
    = \prod_{k=i}^{m} \DR_{ik} \Exp((\accm{}{}^k - \bias_{\acc{}{}}^k - \noise_{\acc{}{}}^k)\dt)
    \label{eq:IMUDeltav}
\end{equation}

The position delta requires more calculations. First, inject equation \eqRef{eq:IMUDeltav} in the position equation of \eqRef{eq:IMUIntim} then rearrange and reorder the terms.

\begin{equation}
\begin{split}
\posi{}{}^{m} - \posi{}{}^{i} &= \sum_{k=i}^{m} \Big[
(\vel{}{}^i + \grav \Dt_{ik} + \Rot{}{}^i \Dv_{ik})\dt 
+ \frac{1}{2}\grav \dt^2 + \frac{1}{2}\Rot{}{}^{k}(\accm{}{}^k - \bias_{\acc{}{}}^k - \noise_{\acc{}{}}^k)\dt^2 \Big]
\\
&= \vel{}{}^i \Dt_{im} + 
\grav\sum_{k=i}^{m} \Dt_{ik} \dt + \frac{1}{2}\grav \sum_{k=i}^{m} \dt^2 +
\Rot{}{}^i\sum_{k=i}^{m} \Big[\Dv_{ik}\dt +  \frac{1}{2} \DR_{ik} (\accm{}{}^k - \bias_{\acc{}{}}^k - \noise_{\acc{}{}}^k)\dt^2 \Big]
\\
&= \vel{}{}^i \Dt_{im} + 
\grav \sum_{k=i}^{m} (\Dt_{ik} \dt + \frac{1}{2}\dt^2) +
\Rot{}{}^i\sum_{k=i}^{m} \Big[\Dv_{ik}\dt +  \frac{1}{2} \DR_{ik} (\accm{}{}^k - \bias_{\acc{}{}}^k - \noise_{\acc{}{}}^k)\dt^2 \Big]
\end{split}
\end{equation}

We will simplify the gravity term that we will call $\cG t_{im}$. Noting that:

\begin{equation*}
    \sum_{k=i}^{m}k = \sum_{k=1}^{m}k - \sum_{k=1}^{i-1}k = \frac{m(m-1)}{2} - \frac{i(i-1)}{2} 
\end{equation*}

we can deduce that:

\begin{equation*}
\begin{split}
\cG t_{im} 
&= \dt \sum_{k=i}^{m}(\Dt_{ik} + \frac{1}{2}\dt) = \dt \sum_{k=i}^{m}((k -i)\dt + \frac{1}{2}\dt)
\\
&= \dt \Big[ \sum_{k=i}^{m} [ k \dt - i(m-i)\dt ] + \frac{1}{2}\Dt_{im}  \Big] 
\\
&= \dt \Big[ \frac{\dt}{2} (m(m-1) - i(i-1) - 2i(m-1)) + \frac{1}{2}\Dt_{im} \Big] 
\\
&= \dt \Big[ \frac{\dt}{2} (i^2 - 2im + m^2 + i - m) + \frac{1}{2}\Dt_{im} \Big] 
\\
&= \dt \Big[ \frac{1}{2} (m-i)^2dt - \frac{1}{2}(m - i)\dt + \frac{1}{2}\Dt_{im} \Big]
\\
&= \frac{1}{2} (m-i)^2dt^2 = \frac{1}{2} \Dt_{im}^2
\end{split}
\end{equation*}
Multiplying by $\Rot{}{}^{i,T}$ and reordering the terms, we can finally define a position delta quantity:

\begin{equation}
    \Dp_{im} \triangleq \Rot{}{}^{i,T}(\posi{}{}^m - \posi{}{}^i - \vel{}{}^i \Dt_{im} - \frac{1}{2} \grav \Dt_{im}^2) = 
    \sum_{k=i}^{m} \Big[\Dv_{ik}\dt +  \frac{1}{2} \DR_{ik} (\accm{}{}^k - \bias_{\acc{}{}}^k - \noise_{\acc{}{}}^k)\dt^2 \Big]
\end{equation}


\section{Elements of the IMU delta matrix Lie group}
\label{sec:IMULieGroup}

\subsection{Tangent space and Lie algebra \texorpdfstring{$\mathfrak{d}$}{d}}
\cite{sola2018micro}
Following \cite{sola2018micro}, the tangent space of $\cD$ at the point $\D$ is found by taking the time derivative of the group constraint, $\D\inv\D = \bfI$.
Noting $\dot{\bullet} \te \dpar{\bullet}{t}$, this yields 
% the tangent space constraint, $\D\inv\dot\D+\dot{(\D\inv)}\D=0$.
% This gives 
after a few manipulations
%
\begin{align}\label{equ:constr}
\D\inv\dot\D 
&=
\begin{bsmallmatrix}
\hatx{\bw} & ~~\DR\tr\bfa~~ & \DR\tr(\bfv-\Dv) \\
\bf0 & 0 & 1 \\
\bf0 & 0 & 0 
\end{bsmallmatrix}
~,
\end{align}
%
with $\bfv \te \dot\Dp$, $\bfa \te \dot\Dv$ and $\hatx{\bw} \te \DR\tr\dot\DR$.
% The equation above defines the tangent space at the point $\D$. 
The Lie algebra $\mathfrak{d}$ is the tangent space at the identity $\D=\bfI$.
Its elements $\bm\nu\hhat 
\te \dot\D |_{\D=\bfI}$ and their isomorphics $\bm\nu$ in Cartesian space are given by, % $\D=\bfI$, so
%
\begin{align}
\bm\nu\hhat 
% \te \dot\D |_{\D=\bfI}
&=
\begin{bsmallmatrix}
\hatx{\bw} & \bfa & \bfv \\
\bf0 & 0 & 1 \\
\bf0 & 0 & 0 
\end{bsmallmatrix} \in \mathfrak{d}
~~~\xrightleftharpoons[\wedge]{~\vee~}~~~ 
\bm\nu=\begin{bsmallmatrix}
\bfv \\ \bfa \\ \bw \\ 1
\end{bsmallmatrix} \in\bbR^{10}
~.
\end{align}
%
This tangent $\bm\nu\hhat$ corresponds to the `velocity' of the group element. 
Any point in the Lie algebra can be obtained after moving at constant velocity during a period $\Dt$, that is, $\bftau\hhat=\bm\nu\hhat\Dt\in\mathfrak{d}$ ---see \eqRef{equ:lie_algebra}.
%
\subsection{The exponential map}
\subsubsection{The general case}
Eq.~\eqRef{equ:constr} can be written as $\dot\D=\D\cdot\bm\nu\hhat$.
This is an ordinary differential equation whose  integral for constant $\bm\nu$ yields the exponential map \cite{sola2018micro}, $ \D(t) = \exp\left(\bm\nu\hhat t\right)$.
This gives a direct expression of the integral of information of the type $(\bfv,\bfa,\bw)$ onto the deltas manifold. See below for the $(\bfa,\bw)$ case.
The closed form of the exponential map is obtained through Taylor expansion (see \eg\ \cite{sola2018micro}\ for examples). 
At $t=\Dt$ we have,
%
\begin{align}
\D(\Dt) 
&= \exp(\bm\nu\hhat\Dt) \te \sum_n \frac1{n!}(\bm\nu\hhat\Dt)^n
~.
\end{align}
%
% with $\bfA=\bftau\hhat\Dt$.
%
Exploiting the cyclic pattern of the powers of $\hatx{\bw}$, this results in
%
\begin{align}
\exp\left(\begin{bsmallmatrix}
\hatx{\bw} & \bfa & \bfv \\
\bf0 & 0 & 1 \\
\bf0 & 0 & 0 
\end{bsmallmatrix}\Dt\right) 
\!=\! \begin{bsmallmatrix}
\exp(\hatx{\bw}\Dt) & \bfQ\bfa\Dt\, & \bfQ\bfv\Dt+\bfP\bfa\Dt^2 \\
\bf0 & 1 & \Dt \\
\bf0 & 0 & 1
\end{bsmallmatrix}
\end{align}
%
with (we skip proofs for space reasons)
%
\begin{align} \label{equ:RQP}
% \bfR (\bth)
%  &= \exp(\hatx{\bth})
%   = \bfI + \sin\theta\hatx{\bfu} + (1-\cos\theta)\hatx{\bfu}^2\\
\bfQ (\bth)
 &= 
  \bfI + \frac{1-\cos\theta}{\theta}\hatx{\bfu} + \frac{\theta-\sin\theta}{\theta}\hatx{\bfu}^2\\
\bfP (\bth)
 &= 
  \frac12\bfI 
   + \frac{\theta-\sin\theta}{\theta^2}\hatx{\bfu} 
   + \frac{\cos\theta + \frac12\theta^2 - 1}{\theta^2}\hatx{\bfu}^2
~,
\end{align}
%
where  $\bth=\bw\Dt$, $\theta=\norm{\bth}$ and $\bfu=\bth/\theta$ form the angle-axis representation of the rotation step $\bw\Dt$. 


\subsubsection{The IMU case of \texorpdfstring{$\bfv=0$}{bfv=0}}
\label{sec:IMU_case}

We defined the IMU deltas as the motion relative to the free-falling frame, which has initial velocity $\bfv_i$. 
Thus the tangent velocity $\bfv=\dot\Dp$  is zero at the start of the integration step. 
%
Since the exponential $\Exp(\bm\nu\Dt)$ assumes a constant tangent vector $\bm\nu=(\bfv,\bfa,\bw,1)$ during the interval $\Dt$, we have that $\bfv=0$ during the full step. 
%
This gives immediately
%
\begin{align}
% \D &= 
\exp\left(\begin{bsmallmatrix}
\hatx{\bw} & \bfa & \bf0 \\
\bf0 & 0 & 1 \\
\bf0 & 0 & 0 
\end{bsmallmatrix}\Dt\right) 
= \begin{bsmallmatrix}
\exp(\hatx{\bw}\Dt) & ~\bfQ\bfa\Dt~ & \bfP\bfa\Dt^2 \\
\bf0 & 1 & \Dt \\
\bf0 & 0 & 1
\end{bsmallmatrix}
~.
\end{align}




\subsection{The adjoint and small adjoint matrices}

%
Following the general methodology explained in \cite{sola2018micro}, the adjoint matrix is obtained by identifying the linear terms in $\Ad{\D}\bftau=(\D\bftau\hhat\D\inv)\vvee$. We get after long but relatively easy calculations,
%
\begin{align}\label{equ:Ad}
\Ad{\D} &=
\begin{bsmallmatrix}
\DR & -\DR\Dt & \hatx{\Dp-\Dv\Dt}\DR & \Dv \\
\bf0 & \DR & \hatx{\Dv}\DR & \bf0 \\
\bf0 & \bf0 & \DR & \bf0 \\
\bf0 & \bf0 & \bf0 & 1
\end{bsmallmatrix} 
\quad
\in\bbR^{10\times10}~.
\end{align}



Similarly, from \cite{EADE-18-DERIVATIVE} the small adjoint matrix can be computed by identifying the linear terms in $
\ad{\bftau}\bfsigma = 
(\bftau\hhat\bfsigma\hhat-\bfsigma\hhat\bftau\hhat)^\vee
$
%
which  for $\bftau=(\bfrho,\bfupsilon,\bth,\Dt)\in\mathfrak{d}$ yields,
%
\begin{align}\label{equ:ad}
\mathrm{\bf ad}_\bftau = \begin{bsmallmatrix}
\hatx{\bth} & -\bfI\Dt & \hatx{\bfrho} & \bfupsilon \\
\bf0 & \hatx{\bth} & \hatx{\bfupsilon} & \bf0 \\
\bf0 & \bf0 & \hatx{\bth} & \bf0 \\
\bf0 & \bf0 & \bf0 & 0 
\end{bsmallmatrix}
\quad
\in\bbR^{10\times10}~.
\end{align}




\subsection{The right Jacobian}

The right Jacobian $\mjac{}{r}$ is the Jacobian of $\Exp()$ as described in \cite{sola2018micro}.
Lacking at the moment a closed form for it, we take the general methodology for the left Jacobian described in \cite{EADE-18-DERIVATIVE}, and transform it to the right using $\mjac{}{r}(\bftau)=\mjac{}{l}(-\bftau)$ \cite{sola2018micro},
%
\begin{align}\label{equ:Jr}
\mjac{}{r}(\bftau) 
= \mjac{}{l}(-\bftau) 
= \sum_i \frac{\ad{-\bftau}^i}{(i+1)!}
= \sum_i \frac{(-\ad{\bftau})^i}{(i+1)!}
% ~
% \in\bbR^{10\times10}
~.
\end{align}
%
This sum can be truncated at the desired degree of accuracy.






\section{Jacobians of force/torque pre-integration}

TODO

