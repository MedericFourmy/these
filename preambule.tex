
%%%%%%%%%%%%%%%%%%%%%%%%%%%%%%%%%%%%%%%%%%%%%%%%%%%%%%%%%%%%%%%%%%%%%


%% Réglage des fontes et typo  
\usepackage{ae,lmodern} % ou seulement l'un, ou l'autre, ou times etc.
\usepackage[utf8]{inputenc}		% LaTeX, comprend les accents !
\usepackage[T1]{fontenc}
\usepackage[english]{babel}

\usepackage[Lenny]{fncychap} 


\usepackage[dvipsnames]{xcolor}  % Coloured text etc.
\usepackage{graphicx}
\usepackage{mathrsfs}

\usepackage{listings}
\definecolor{gray}{rgb}{0.4,0.4,0.4}
\definecolor{darkblue}{rgb}{0.0,0.0,0.6}
\definecolor{cyan}{rgb}{0.0,0.6,0.6}
\lstset{
  basicstyle=\ttfamily,
  columns=fullflexible,
  showstringspaces=false,
  commentstyle=\color{gray}\upshape
}
\lstdefinelanguage{XML}
{
  morestring=[b]",
  morestring=[s]{>}{<},
  morecomment=[s]{<?}{?>},
  stringstyle=\color{black},
  identifierstyle=\color{darkblue},
  keywordstyle=\color{cyan},
  morekeywords={diameter, name}% list your attributes here
}
\lstset{language=XML}

%%%%%%%%%%%%%%%%%%%%%%%%%%%%%%%%%%%%%%%%%%%%%%%%%%%%%%%%%%%%%%%%%%%%%

%% Apparence globale             
\usepackage[top=3cm, bottom=2cm, left=3cm, right=3cm, headheight=15pt]{geometry} 

% Header footer
% E for even page
% O for odd page
% L for left side
% C for centered
% R for right side
\usepackage{fancyhdr}	
	\pagestyle{fancy}
  \fancyhf{} %clears the header and footer, otherwise the elements of the default "plain" page style will appear
  \fancyfoot[C]{\thepage}
  \fancyhead[LE]{Chapter \thechapter: \leftmark}
  \fancyhead[RO]{\rightmark}
  
  \renewcommand{\headrulewidth}{1pt}
  % \renewcommand{\footrulewidth}{1pt}

\renewcommand{\sectionmark}[1]{\markright{\thesection.\ #1}}
\renewcommand{\chaptermark}[1]{%
  \markboth{\if@mainmatter\@chapapp\ \thechapter. \fi\ #1}
  {\if@mainmatter\@chapapp\ \thechapter. \fi\ #1}} 


\usepackage{enumerate}
\usepackage{enumitem}
\usepackage[section]{placeins}	% Place un FloatBarrier à chaque nouvelle section
\usepackage{epigraph}
\usepackage[font={small}]{caption}
\usepackage[english]{minitoc}		% Mini table des matières, en français
	\setcounter{minitocdepth}{2}	% Mini-toc détaillées (sections/sous-sections)
\usepackage{pdflscape}				% Permet d'utiliser des pages au format paysage


%%%%%%%%%%%%%%%%%%%%%%%%%%%%%%%%%%%%%%%%%%%%%%%%%%%%%%%%%%%%%%%%%%%%%
% Biblio                        
%\makeatletter
%\patchcmd{\BR@backref}{\newblock}{\newblock(page~}{}{}	% Pour les back-references, affiche 'page' au lieu de 'p.'
%\patchcmd{\BR@backref}{\par}{)\par}{}{}
%\makeatother
	

%%%%%%%%%%%%%%%%%%%%%%%%%%%%%%%%%%%%%%%%%%%%%%%%%%%%%%%%%%%%%%%%%%%%%
% Tableau
\usepackage{array}
\usepackage{hhline}
\usepackage{adjustbox}
\usepackage{multirow,makecell}
\usepackage{color}
\usepackage{arydshln}

% Figures
% subcaption provides subfigure & subtable environments and \subcaptionbox command (2 equivalent ways to proceed)
% subfigure env is based on the minipage
% \subcaptionbox is based on \parbox
% "Please note: This package is incompatible with the subfigure and subfig packages." from CTAN doc
\usepackage{subcaption}

% Formatting of tabular cells
\usepackage{makecell}


% Algo
\usepackage{algorithm}
\usepackage{algorithmic}

%%%%%%%%%%%%%%%%%%%%%%%%%%%%%%%%%%%%%%%%%%%%%%%%%%%%%%%%%%%%%%%%%%%%%
%% Mise en forme du texte        
\usepackage{xspace}
%\usepackage[load-configurations = abbreviations]{siunitx}
%	\DeclareSIUnit{\MPa}{\mega\pascal}
%	\DeclareSIUnit{\micron}{\micro\meter}
%	\DeclareSIUnit{\tr}{tr}
%	\DeclareSIPostPower\totheM{m}
%	\sisetup{
%	locale = FR,
%	  inter-unit-separator=$\cdot$,
%	  range-phrase=~\`{a}~,     	% Utilise le tiret court pour dire "de... à"
%	  range-units=single,  			% Cache l'unité sur la première borne
%	  }
%\usepackage{chemist}
%\usepackage[version=3]{mhchem}
\usepackage{textcomp}
\usepackage{numprint}

\usepackage{hyphenat}


\usepackage{times}
\usepackage{amsmath,amssymb,mathtools}
\usepackage{tikz-uml}
\usepackage{environ}

% to be able to scale tikzpicture
\makeatletter
\newsavebox{\measure@tikzpicture}
\NewEnviron{scaletikzpicturetowidth}[1]{%
  \def\tikz@width{#1}%
  \def\tikzscale{1}\begin{lrbox}{\measure@tikzpicture}%
  \BODY
  \end{lrbox}%
  \pgfmathparse{#1/\wd\measure@tikzpicture}%
  \edef\tikzscale{\pgfmathresult}%
  \BODY
}
\makeatother

%%%%%%%%%%%%%%%%%%%%%%%%%%%%%%%%%%%%%%%%%%%%%%%%%%%%%%%%%%%%%%%%%%%%%
%% Pour changer localement les marges

\newenvironment{changemargin}[2]{%
\begin{list}{}{%
\setlength{\leftmargin}{#1}%
\setlength{\rightmargin}{#2}%
\setlength{\listparindent}{\parindent}%
\setlength{\itemindent}{\parindent}%
\setlength{\parsep}{\parskip}%
}%
\item[]}{\end{list}}

\definecolor{ieeeblue}{RGB}{0,98,155}

\usepackage{bm}


\usepackage[sorting=none,
  style=numeric,
  backref=true,
  backend=biber,
  maxnames=6,
  minnames=1]{biblatex}

% \usepackage[
% backend=biber,
% style=numeric,
% ]{biblatex}
% recommended when using biblatex with babel
\usepackage{csquotes}

% \usepackage{varioref}       % to ref "far away" labels

% Create interal document links
\usepackage[pdftex,
  pdfborder={0 0 0},
  colorlinks=true,
  linkcolor=blue,
  citecolor=red,
  pagebackref=false,
  ]{hyperref} 

% \usepackage{cleveref}  % use \cref instead of \ref   --  BUG   


% TODO notes
\usepackage{xargs}                      % Use more than one optional parameter in a new commands
\usepackage[dvipsnames]{xcolor}  % Coloured text etc.
% 
\usepackage[colorinlistoftodos,prependcaption,textsize=tiny]{todonotes}
\newcommandx{\unsure}[2][1=]{\todo[linecolor=red,backgroundcolor=red!25,bordercolor=red,#1]{#2}}
\newcommandx{\change}[2][1=]{\todo[linecolor=blue,backgroundcolor=blue!25,bordercolor=blue,#1]{#2}}
\newcommandx{\info}[2][1=]{\todo[linecolor=OliveGreen,backgroundcolor=OliveGreen!25,bordercolor=OliveGreen,#1]{#2}}
\newcommandx{\improvement}[2][1=]{\todo[linecolor=Plum,backgroundcolor=Plum!25,bordercolor=Plum,#1]{#2}}
\newcommandx{\thiswillnotshow}[2][1=]{\todo[disable,#1]{#2}}


% include all bib files separately.
% to use wildecard upgrade biber to 2.13
\addbibresource{bibfiles/these.bib}
\addbibresource{bibfiles/applications.bib}
\addbibresource{bibfiles/centroidal-est.bib}
\addbibresource{bibfiles/contact-detection.bib}
\addbibresource{bibfiles/exteroceptive.bib}
\addbibresource{bibfiles/factor-graph-est.bib}
\addbibresource{bibfiles/improving-kin.bib}
\addbibresource{bibfiles/object-level.bib}
\addbibresource{bibfiles/online-smoothing.bib}
\addbibresource{bibfiles/planning.bib}
\addbibresource{bibfiles/preintegration.bib}
\addbibresource{bibfiles/proprioceptive-est.bib}


\usepackage{customCommands}
\usepackage{customCommandsBis}
