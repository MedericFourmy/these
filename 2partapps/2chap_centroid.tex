\chapter{Centroidal estimation}
\label{chp:centroidal_estimation}
\minitoc
\bigskip


In this chapter, we present work published in our paper \cite{fourmy2021contact}.

- Prelaminary work on Inertial Kinematics estimation
- Centroidal estimation results

\section{Introduction}


In this section we present results of the application of the estimator on a dataset taken at LAAS-CNRS on the open source robot Solo-12 
\cite{grimminger2020open} that we recently built:
\textit{sinXYZ} corresponds to a trajectory where the robot moves by following a sine wave reference with the feet fixed in place. 
% This movement and another one are displayed in the companion video \url{https://peertube.laas.fr/videos/watch/16822d27-3557-4e35-9a0d-ce5b0aea4c27}.
%
We use the state estimation framework WOLF, developed at IRI-Barcelona, which relies on Google Ceres \cite{ceres-solver} as the graph solver. 
The robot kinematics and dynamics computations are handled by the Pinocchio library \cite{carpentier2015pinocchio}. For all runs, KFs were created 
at 4Hz for fair comparison.


\section{Experimental setup}
%  
As is often the case with quadruped robots, Solo-12 is not equipped with three axis force sensors at its feet. 
Yet, in order to validate the present method, it is possible to reconstruct the contact forces based on the robot dynamics 
equation \eqref{eq:wbdyn}. Knowing the robot configuration and derivatives $\q,\vq,\dvq$ and joint torques $\bm\tau$ (from motor currents), 
recovering forces from this equation results in solving an over-determined linear system. Some of these quantities are hard to obtain directly 
since they depend on the state being estimated (\eg base orientation) or on numerical differentiation ($\ddqa$). For these reasons, we pre-calculated 
these forces by benefiting from an internal filter of the 3DM-CX5-25 IMU for the base orientation, centered window differentiation of encoder speed 
measurements for the articulation acceleration $\ddqa$ and neglecting the influence of linear velocity. Fig. \ref{fig:force_est} shows an example of 
the force reconstruction of one leg using the robot proprioceptive sensors.
%
\begin{figure}
    \centering
    \includegraphics[width=0.9\columnwidth]{figures/centroidal/forces_solo_1leg.pdf}
    \caption{Force estimation on X-Y-Z axis (r-g-b) of one Solo-12 leg expressed in world frame using proprioceptive sensors during the \textit{sinXYZ} trajectory}
    \label{fig:force_est}
\end{figure}

\subsection{Base estimation through inertial kinematic fusion}
First, to validate the use of our kinematic factor, we include uniquely the IMU and LO factors to obtain an Inertial Kinematics  estimator which 
conceptually includes the same information as estimators such as \cite{bloesch2013state}. In Fig.~\ref{fig:PosiIK4}, we compare our state estimation 
at 1kHz with motion capture (Mo-Cap) up-sampled from 200Hz to 1kHz.
Velocity in base frame is also shown in Fig.~\ref{fig:VelIK4}.
%
\begin{figure}[t]
    \centering
    \includegraphics[height=0.6\columnwidth]{figures/centroidal/base_position_IK4.pdf}
    \caption{\textit{sinXYZ} trajectory base position from the IMU+Kinematics (IK) estimator (blue) vs Mo-Cap (red)}
    \label{fig:PosiIK4}
\end{figure}
%
\begin{figure}[t]
    \centering
    \includegraphics[height=0.6\columnwidth]{figures/centroidal/base_velocity_base_frame_IK4.pdf}
    \caption{\textit{sinXYZ} trajectory base velocity in base frame from the IMU+kinematics estimator (blue) vs Mo-Cap (red)}
    \label{fig:VelIK4}
\end{figure}

Artificially removing contact factors (considering only 1, 2 or 3 feet in contact) can help us gain confidence in the use of this kinematic factor 
in situations where we rarely have all feet in contact, like for example with trotting gaits. In Fig.~\ref{fig:ErrIKn}, we can see that only considering 
1 foot in contact during the whole trajectory results in a drifting position, but as soon as 2 or more feet are in contact, the system is constrained enough 
for the drift to remain below around 5mm on all axes. 
%
\begin{figure}[t]
    \centering
    \includegraphics[height=0.6\columnwidth]{figures/centroidal/base_position_err_IKn.pdf}
    \caption{\textit{sinXYZ} trajectory base position error with different numbers of feet used for the leg odometry factors: 
    1 (blue), 2 (orange), 3 (green), 4 (red) from the IMU+kinematics estimator}
    \label{fig:ErrIKn}
\end{figure}
%

\subsection{Centroidal estimation}

Now, on the same trajectory, we deploy the full estimator with all factors described in Section \ref{sec:factor-graph} to jointly 
estimate the base and centroidal quantities. A ground truth on the centroidal quantities is difficult to obtain since no direct 
sensor can provide us with this information contrary to the base state. 
We can however validate our method by comparing it to a two step procedure: first, estimate the base state with a state Kalman filter as 
implemented in \cite{bledt2018cheetah}, then compute the centroidal quantities directly from the robot kinematic model. 
The full estimator should be able to infer a bias on the $\posim{B}{C}$ measure so we artificially add a constant disturbance 
in the robot dynamic model on the lever of the base link of [0.03, 0.06, 0.04]\,cm, which then corresponds to a CoM bias of 
[-0.0197, -0.0394,  0.0263] cm. Fig.~\ref{fig:bias_est} shows that the bias estimated with our method closely matches the introduced bias. 
Fig.~\ref{fig:comparison_KF_wolf} shows a comparison between the base and CoM reconstruction with our method and with the two-step base Kalman filter 
with geometric CoM. Note that the base-CoM difference on the z axis reflects the fact that limbs of the robot naturally lowers  its CoM. 
The estimated CoM velocity closely follows the velocity of the base as shown in Fig.~\ref{fig:com_base_vel}.



% \begin{figure}[t]
%     \centering
%     \includegraphics[height=0.6\columnwidth]{figures/centroidal/com_position_povcdl_sin.pdf}
%     \caption{Base position (blue) vs CoM (red) from the full estimator}
%     \label{default}
% \end{figure}

\begin{figure}[t]
    \centering
    \includegraphics[height=0.6\columnwidth]{figures/centroidal/com_velocity_povcdl_sin.pdf}
    \caption{Base (blue) vs CoM (red) velocities   from the full estimator}
    \label{fig:com_base_vel}
\end{figure}


\begin{figure}[t]
    \centering
    \includegraphics[width=0.8\columnwidth]{figures/centroidal/com_bias_est.pdf}
    \caption{Estimation of bias on CoM measurement from the full estimator along x-y-z axis ( red-green-blue) in base frame}
    \label{fig:bias_est}
\end{figure}

\begin{figure}[t]
    \centering
    \includegraphics[height=0.6\columnwidth]{figures/centroidal/base_com_position_wolf_vs_KF.pdf}
    \caption{Comparison of the estimates between decoupled the Kalman filter base estimator and geometric CoM reconstruction (blue) and 
    the tightly coupled estimator presented in this paper (red) on the \textit{sinXYZ} trajectory with artificial base link CoM bias.}
    \label{fig:comparison_KF_wolf}
\end{figure}


